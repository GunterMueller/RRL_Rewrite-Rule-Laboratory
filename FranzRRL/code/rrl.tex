\documentstyle[11pt]{article}
\parskip 5pt
\topmargin -0.3in
\textheight 9in
\textwidth 6.2in
\oddsidemargin=0.2in
\evensidemargin=0.2in
\newcommand{\Grobner}{Gr\"{o}bner\ }
\newcommand{\Groebner}{Gr\"{o}bner\ }
\newcommand{\RRL}{{\sl RRL}\ }
\newcommand{\ERRL}{{\sl RRL}}
\pagestyle{empty}
\begin{document}
\title{An Overview of Rewrite Rule Laboratory (RRL)\thanks
{Partially supported by the National Science Foundation Grant
no. CCR-8408461.}}
\author{{\bf Deepak Kapur} \\
Department of Computer Science \\
State University of New York at Albany \\
Albany, NY 12222 \\
kapur@albanycs.albany.edu \\
\and
{\bf Hantao Zhang} \\
Department of Computer Science \\
The University of Iowa \\
Iowa City, IA 52242 \\
hzhang@herky.cs.uiowa.edu}

\date{}
\maketitle

\section{What is \ERRL?}

{\sl Rewrite Rule Laboratory} (\ERRL) is a theorem proving environment
for experimenting with existing automated reasoning algorithms for
equational logic based on rewriting techniques as well as for
developing new reasoning algorithms
\cite{KapurSiva,KZ87,KZ881}.  Reasoning
algorithms are used in many application areas in computer science and
artificial intelligence, in particular, analysis of specifications of
abstract data types, verification of hardware and software, synthesis
of programs from specifications, automated theorem proving, expert
systems, deductive data basis, planning and recognition, logic
programming, constraint satisfaction, natural language processing,
knowledge-based systems, robotics, vision.  

The project for developing rewrite rule laboratories was jointly
conceived in 1983 by the first author and David Musser at Rensselaer
Polytechnic Institute (RPI) and John Guttag of MIT when rewrite
methods were gaining attention because of their use in the analysis of
algebraic specifications of user-defined data types.  The work on the
\RRL project was started at General Electric Corporate Research and
Development (GECRD) and RPI in Fall 1983, following a workshop on
rewrite rule laboratories in Schenectady, New York
\cite{Guttagetal84}.  Since then, \RRL has continued to evolve.
\RRL is written in a subset of Zeta Lisp that is (almost) compatible
with Franz Lisp and is more than 15,000 lines of code.  In order to
make \RRL portable and available for wider distribution, conscious
attempt was made not to use special features of Zeta Lisp and the Lisp
machine environment.
 
\RRL was initially used to study existing algorithms, including the
Knuth-Bendix completion procedure \cite{KnuthBendix}, and
Dershowitz's algorithm for recursive path ordering \cite{Dershowitz}
to orient equations into terminating rewrite
rules \cite{KapurSiva}.  However, this environment soon began serving as a
vehicle for generating new ideas, concepts and algorithms which could
be implemented and tested for their effectiveness.  Experimentation
using \RRL has led to development of new heuristics for decision
procedures and semi-decision procedures in theorem proving
\cite{KMN84,ZK882}, complexity studies of primitive operations
used in \RRL \cite{KN871}, and new approaches for first-order theorem
proving \cite{KN85,ZK881}, proofs by induction
\cite{KNZ861,ZKM}.

\RRL has been distributed to over 20 universities and research
laboratories. Any one interested in obtaining a copy of \RRL can write
to the authors.

\section{What can \RRL do?}

\RRL currently
provides facilities for
\begin{itemize}
\item 
automatically proving theorems in first-order predicate calculus
with equality,
\item
generating decision procedures for first-order (equational)
theories using completion procedures,
\item 
proving theorems by 
induction using different approaches - methods
based on the {\sl proof by consistency} approach \cite{Musser80,KM84}
(also called the {\sl inductionless-induction} approach) as well as 
the explicit induction approach, and
\item 
checking the consistency and completeness of equational
specifications.
\end{itemize}

The input to \RRL is a first-order theory specified by a finite set of
axioms.  Every axiom is first transformed into an equation (or a
conditional equation).  The kernel of \RRL is the {\sl extended
Knuth-Bendix completion procedure} \cite{KnuthBendix,LB77,PS81} which
first produces terminating rewrite rules from equations and then
attempts to generate a canonical rewriting system for the theory
specified by a finite set of formulas.  If \RRL is successful in
generating a canonical set of rules, these rules associate a unique
normal form for each congruence class in the congruence relation
induced by the theory.  These rules thus serve as a decision procedure
for the theory.  

For proving a single first-order formula, \RRL also supports the
proof-by-contradiction (refutational) method. The set of hypotheses
and the negation of the conclusion are given as the input, and \RRL
attempts to generate a contradiction, which is a system including the
rule $true \rightarrow false$.

New algorithms and methods for automated reasoning have been developed
and implemented in \ERRL.  \RRL has two different methods implementing
the proof by consistency approach for proving
properties by induction: the {\sl test set} method developed in
\cite{KNZ861} as well as the method using {\sl quasi-reducibility}
developed in \cite{JK86}.  These methods can be used for checking
structural properties of equational specifications such as the {\sl
sufficient-completeness} \cite{KNZ85} and consistency properties
\cite{Musser80,KM84}.
 
Recently, a method based on the concept of a {\it cover set} of a
function definition has been developed for mechanizing proofs by
explicit induction for equational specifications \cite{ZKM}.  The
method is closely related to Boyer and Moore's approach.  It
has been successfully used to prove over a hundred theorems about
lists, sequences, and number theory including the unique prime
factorization theorem for numbers; see \cite{ZKM}.
 
\RRL supports an implementation of a first-order theorem proving method
based on polynomial representation of formulas using the exclusive-or
connective as `+' and the and connective as `.' \cite{Hsiang85}, and
using \Groebner basis like completion procedure \cite{KN85}.  Its
performance compares well with theorem provers based on resolution
developed at Argonne National Lab as well as theorem provers based on
the connection graph approach developed at SRI International and University of
Kaiserslautern. Most of the hardware verification work mentioned later
in the section on applications of \RRL was done using this method of
theorem proving.
 
Recently, another method for automatically proving theorems in
first-order predicate calculus with equality using conditional
rewriting has also been implemented and successfully tried on a number
of problems \cite{ZK881}.  This method combines ordered resolution and
oriented paramodulation of Hsiang and Rusinowitch \cite{HR87} into a
single inference rule, called {\it clausal superposition}, on
conditional rewrite rules. These rewrite rules can be used to simplify clauses
using conditional rewriting.
 
\RRL supports different strategies for normalization and generation of
critical pairs and other features. These strategies play a role
similar to the various strategies and heuristics available in
resolution-based theorem provers. \RRL provides many options to the
user to adapt it to a particular application domain. Because of
numerous options and strategies supported by \ERRL, it can be used as a
laboratory for testing algorithms and strategies. 

\section{Some Applications of \RRL}
 
\RRL has been successfully used to attack difficult problems
considered a challenge for theorem provers.  An automatic proof of a
theorem that an associative ring in which every element $x$ satisfies
$x^3 = x,$ is commutative, was obtained using \RRL in nearly 2 minutes
on a Symbolics machine.  Previous attempts to solve this problem using
the completion procedure based approach required over 10 hours of
computing time \cite{Stickel84}.  With some special heuristics
developed for the family of problems that for $n > 1$, for every $x$,
$x^n = x$ implies that an associative ring is commutative, \RRL takes
5 seconds for the case when $n$ = 3, 70 seconds for $n$ = 4, and 270
seconds for $n$ = 6 \cite{KMN84,ZK882}.  To our knowledge, this is the
first computer proof of this theorem for $n$ = 6. Using algebraic
techniques, we have been able to prove this theorem for many even
values of $n$. \RRL has also been used to prove the equivalence of
different non-classical axiomatizations of free groups developed by
Higman and Neumann to the classical three-law characterization
\cite{KZ882}.
 
\RRL can be also used as a tool for demonstrating approaches towards
specification and verification of hardware and software.  \RRL is a
full-fledged theorem prover supporting different methods for
first-order and inductive theorem proving in a single reasoning
system.  It has been successfully used at GECRD for hardware
verification. Small combinational and sequential circuits, leaf cells
of a bit-serial compiler have been verified \cite{NS881}. The most
noteworthy achievement has been the detection of 2 bugs in a VHDL
description of a 10,000 transistor chip implementing Sobel's edge
detection algorithm for image processing and after fixing those bugs,
verifying the chip description to be consistent with its behavioral
description \cite{NS882}.

\RRL has also been used in teaching a course on theorem proving
methods based on rewrite techniques, 
providing students with an opportunity to try methods and
approaches discussed in the course.  It has served as a basis for
class projects and thesis research. 

Additional details about the capabilities of \RRL and the associated theory
as well as possible applications of \RRL can be found in the 
papers listed in the selected bibliography at the end of this write-up.

\section{Future Plans}

We plan to extend \RRL to make it more useful and convenient for
reasoning about software and hardware.  The test set and cover set methods
for supporting automatic proofs by induction in \RRL will be further
developed and integrated, with a particular emphasis on the cover set
method.  Approaches for reasoning with incomplete specifications will
also be investigated.

In addition, theoretical and experimental research in developing
heuristics, including identifying unnecessary computations for
completion procedures as well as for first-order theorem proving will
be undertaken to improve their performance.  Complexity studies and
efficient implementations of primitive operations will also be
continued and results will be incorporated into \ERRL.

\begin{thebibliography}{99}

\bibitem{Dershowitz}Dershowitz, N. (1987). Termination of rewriting.
{\sl J. of Symbolic Computation}.

\bibitem{Guttagetal84}Guttag, J.V., Kapur, D., Musser, D.R. (eds.)
(1984).  Proceedings of an {\sl NSF Workshop on the Rewrite Rule
Laboratory} Sept. 6-9 Sept. 1983, General Electric Research and
Development Center Report 84GEN008.

\bibitem{Hsiang85}Hsiang, J. (1985). Refutational theorem proving
using term-rewriting systems. {\sl Artificial Intelligence} Journal,
25, 255-300.

\bibitem{HR87} Hsiang, J., and Rusinowitch, M. (1986). ``A new method for 
establishing refutational completeness in theorem proving,''
{\sl Proc. 8th Conf. on Automated Deduction},
LNCS No. 230, Springer Verlag, 141-152.

\bibitem{JK86}Jouannaud, J., and Kounalis, E. (1986).
Automatic proofs by induction in equational theories without
constructors.  Proc. of {\sl Symposium on Logic in Computer Science},
358-366.

\bibitem{KM84}Kapur, D., and Musser, D.R. (1984). Proof by
consistency.  In Reference \cite{Guttagetal84}, 245-267.
Also in the {\it
Artificial Intelligence} Journal, Vol. 31, Feb. 1987, 125-157. 

\bibitem{KMN84}Kapur, D., Musser, D.R., and Narendran, P. (1984). Only
prime superpositions need be considered for the Knuth-Bendix
completion procedure.  GE Corporate Research
and Development Report, Schenectady.  Also in {\it Journal of Symbolic Computation} Vol. 4, August 1988.
 
\bibitem{KN85}Kapur, D., and Narendran, P. (1985).
An equational approach to theorem proving
in first-order predicate calculus. Proc. of {\sl 8th IJCAI},
Los Angeles, Calif.

\bibitem{KN871} Kapur, D., and Narendran, P. (1987). Matching, Unification and
Complexity. {\it SIGSAM Bulletin}.

\bibitem{KNZ85}Kapur, D., Narendran, P., and Zhang, H (1985). On
sufficient completeness and related properties of term rewriting
systems.  GE Corporate Research and Development Report, Schenectady,
NY.  {\it Acta Informatica,} Vol. 24, Fasc. 4, August 1987, 395-416.

\bibitem{KNZ861}Kapur, D., Narendran, P., and Zhang, H. (1986). Proof
by induction using test sets. {\sl
Eighth International Conference on
Automated Deduction} (CADE-8), Oxford, England, July 1986, 
Lecture Notes in Computer Science, 230, Springer Verlag, 
New York, 99-117.

\bibitem{KapurSiva}Kapur, D. and Sivakumar, G. (1984) Architecture of
and experiments with RRL, a Rewrite Rule Laboratory. In: Reference
\cite{Guttagetal84}, 33-56.

\bibitem{KZ87} Kapur, D., and Zhang, H. (1987). {\it RRL: A Rewrite Rule Laboratory -
User's Manual.} GE Corporate Research and Development Report,
Schenectady, NY, April 1987.

\bibitem{KZ881} Kapur, D., and Zhang, H. (1988).
{\sl RRL: A Rewrite Rule Laboratory}.  Proc. of {\it Ninth
International Conference on Automated Deduction} (CADE-9), Argonne,
IL, May 1988.

\bibitem{KZ882} Kapur, D., and Zhang, H. (1988). Proving equivalence of 
different axiomatizations of free groups.  {\it J. of Automated
Reasoning} 4, 3, 331-352.

\bibitem{KnuthBendix}Knuth, D.E. and Bendix, P.B. (1970). Simple word
problems in universal algebras.  In: {\sl Computational Problems in
Abstract Algebras.} (ed. J.  Leech), Pergamon Press, 263-297.

\bibitem{LB77}Lankford, D.S., and Ballantyne, A.M.
(1977). Decision procedures for simple equational theories with
commutative-associative axioms: complete sets of
commutative-associative reductions.
Dept. of Math. and Computer Science, University of Texas, Austin,
Texas, Report ATP-39.

\bibitem{Musser80}Musser, D.R. (1980). On proving inductive
properties of abstract data types. Proc. {\sl 7th Principles of
Programming Languages (POPL)}.

\bibitem{NS881} Narendran, P., and Stillman, J. (1988). Hardware verification in the 
Interactive VHDL Workstation. In: {\sl VLSI Specification,
Verification and Synthesis } (eds. G. Birtwistle and P.A.
Subrahmanyam), Kluwer Academic Publishers, 217-235.

\bibitem{NS882} Narendran, P., and Stillman, J. (1988).
Formal verification of the Sobel image processing chip.  GE Corporate
Research and Development Report, Schenectady, NY, November 1987. Proc.
{\sl Design Automation Conference}.

\bibitem{PS81}Peterson, G.L., and Stickel, M.E. (1981).
Complete sets of reductions for some equational theories. {\sl J.
ACM}, 28/2, 233-264.

\bibitem{Stickel84}
Stickel, M.E. (1984). ``A case study of theorem proving by the
Knuth-Bendix method: discovering that $x^3 = x$ implies ring
commutativity'', Proc. of 7th Conf. on Automated Deduction,
Springer-Verlag LNCS 170, pp. 248-258.

\bibitem{ZK881} Zhang, H., and Kapur, D. (1987). First-order
theorem proving using conditional rewriting. 
Proc. of {\it Ninth International
Conference on Automated Deduction} (CADE-9), Argonne, IL, May 1988.

\bibitem{ZK882} Zhang, H., and Kapur, D. (1989). Consider
only general superpositions in completion procedures. {\it This
Proceedings}.

\bibitem{ZKM} Zhang, H., Kapur, D., and Krishnamoorthy,
M.S.  (1988). A mechanizable induction principle for equational
specifications.  Proc.
of {\it Ninth International Conference on Automated Deduction}
(CADE-9), Argonne, IL, May 1988.

\end{thebibliography}

\end{document}



