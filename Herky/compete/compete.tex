\documentstyle[11pt]{article}
\topmargin -0.3in
\textheight 8.8in
\textwidth 6.5in
\parskip 6pt
\oddsidemargin=0.0in
\evensidemargin=0.0in
\begin{document}

\begin{center}
{\LARGE\bf 			 A Proposal for a Competition } \\[.15in]
\Large\bf Ross Overbeek \\[.05in]
\Large\em Mathematics and Computer Science Division \\
\Large\em Argonne National Laboratory\\[.2in]
June 1, 1990
\end{center}


In the past, people have discussed the idea of starting theorem-proving
competitions, but we were never able to formulate an acceptable mechanism
for comparing different systems.  I propose that we have a number of small
competitions on constrained aspects of automated deduction, with no pretense
that the winning sytems will have to be ``best'' in any overall sense.  The
notion of such competitions would be to allow researchers to focus on
well-defined, relevant issues and investigate alternative technologies in a
limited context.

To this end, I offer two \$1000 prizes -- one for a competition in basic
components of resolution theorem-proving and one for problems based on
equality.   Here are the details of the two competitions:


\section*{Basic Components of Resolution Theorem Proving}

This competition will be to produce proofs of the following seven problems
(the first five appeared in ``Subsumption, a Sometimes Undervalued Procedure'' by
Wos, Overbeek, and Lusk).

\subsection*{Theorems:}
\begin{enumerate}
\item  In a group, if (for all elements $x$) the square of $x$ is the identity $e$, then 
    the group is commutative.

\item  In a group, if (for all $x$) the cube of $x$ is the identity $e$, then the 
    equation $[[x,y],y]=e$ holds, where $[x,y]$ is the product of $x$, $y$, 
    the inverse of $x$, and the inverse of $y$ (i.e., $[x,y]$ is the commutator of
    $x$ and $y$).

\item  In a ring, if (for all $x$) the cube of $x$ is $x$, then the ring is commutative.

\item  The formula $XGK = e(x,e(e(y,e(z,x)),e(z,y))))$ is a shortest single axiom 
    for the equivalential calculus.

\item  The formula $i(i(i(x,y),z),i(i(z,x),i(u,x)))$ is a shortest single axiom 
    of the implicational calculus.

\item  Two theorems from a calculus closely related to the implicational calculus
    (think of $i(x,y)$ as ``$x$ implies $y$'' and $n(x)$ as ``the negation of $x$'').
\end{enumerate}

\subsection*{Clauses for each problem:}

\begin{verbatim}
Theorem 1:   
  
    % P(x,y,z) means that the product of x and y is z.

    % e is a two-sided identity
    P(e,x,x).
    P(x,e,x).
  
    % g(x) is a two-sided inverse for x
    P(g(x),x,e).
    P(x,g(x),e).
  
    % closure
    P(x,y,f(x,y)).
  
    % associativity
    -P(x,y,u) | -P(y,z,v) | -P(u,z,w) | P(x,v,w).
    -P(x,y,u) | -P(y,z,v) | -P(x,v,w) | P(u,z,w).
  
    % product is well-defined
    -P(x,y,u) | -P(x,y,v) | eq(u,v).
  
    % equality axioms
    eq(x,x).
    -eq(x,y) | eq(y,x).
    -eq(x,y) | -eq(y,z) | eq(x,z).
  
    % equality substitution axioms
    -eq(u,v) | -P(u,x,y) | P(v,x,y).
    -eq(u,v) | -P(x,u,y) | P(x,v,y).
    -eq(u,v) | -P(x,y,u) | P(x,y,v).
    -eq(u,v) | eq(f(u,x),f(v,x)).
    -eq(u,v) | eq(f(x,u),f(x,v)).
    -eq(u,v) | eq(g(u),g(v)).
  
      % the square of every x is the identity
    P(x,x,e).
    
    % Denial: there exist two elements that do not commute
    P(a,b,c).
    -P(b,a,c).
  
Theorem 2:
  
    % e is a two-sided identity
    P(e,x,x).
    P(x,e,x).
  
    % g(x) is a two-sided inverse for x
    P(g(x),x,e).
    P(x,g(x),e).
  
    % closure
    P(x,y,f(x,y)).
  
    %associativity
    -P(x,y,u) | -P(y,z,v) | -P(u,z,w) | P(x,v,w).
    -P(x,y,u) | -P(y,z,v) | -P(x,v,w) | P(u,z,w).
  
    % product is well-defined
    -P(x,y,u) | -P(x,y,v) | eq(u,v).
  
    % equality axioms
    eq(x,x).
    -eq(x,y) | eq(y,x).
    -eq(x,y) | -eq(y,z) | eq(x,z).
  
    % equality substitution axioms
    -eq(u,v) | -P(u,x,y) | P(v,x,y).
    -eq(u,v) | -P(x,u,y) | P(x,v,y).
    -eq(u,v) | -P(x,y,u) | P(x,y,v).
    -eq(u,v) | eq(f(u,x),f(v,x)).
    -eq(u,v) | eq(f(x,u),f(x,v)).
    -eq(u,v) | eq(g(u),g(v)).
  
    % x cubed is e
    -P(x,x,y) | P(x,y,e).
    -P(x,x,y) | P(y,x,e).

    % Denial: for some a and b, [[a,b],b] is not e
    P(a,b,c).
    P(c,g(a),d).
    P(d,g(b),h).
    P(h,b,j).
    P(j,g(h),k).
    -P(k,g(b),e).
  
Theorem 3:
  
    % 0 is an additive identity 
    S(0,x,x).
    S(x,0,x).
  
    % g(x) is an additive inverse for x
    S(g(x),x,0).
    S(x,g(x),0).
  
    % closure of addition
    S(x,y,j(x,y)).
  
    % associativity of addition
    -S(x,y,u) | -S(y,z,v) | -S(u,z,w) | S(x,v,w).
    -S(x,y,u) | -S(y,z,v) | -S(x,v,w) | S(u,z,w).
    
    % addition is well-defined
    -S(x,y,u) | -S(x,y,v) | eq(u,v).
  
    % equality axioms    
    eq(x,x).
    -eq(x,y) | eq(y,x).
    -eq(x,y) | -eq(y,z) | eq(x,z).
  
    % equality substitution axioms
    -eq(u,v) | -S(u,x,y) | S(v,x,y).
    -eq(u,v) | -S(x,u,y) | S(x,v,y).
    -eq(u,v) | -S(x,y,u) | S(x,y,v).
    -eq(u,v) |  eq(j(u,x),j(v,x)).
    -eq(u,v) |  eq(j(x,u),j(x,v)).
    -eq(u,v) |  eq(g(u),g(v)).
  
    % commutativity of addition
    -S(x,y,z) | S(y,x,z).
  
    % multiplication by 0
    P(0,x,0).
    P(x,0,0).
  
    % closure for multiplication
    P(x,y,f(x,y)).
  
    % associativity of multiplication
    -P(x,y,u) | -P(y,z,v) | -P(u,z,w) | P(x,v,w).
    -P(x,y,u) | -P(y,z,v) | -P(x,v,w) | P(u,z,w).
  
    % distributive laws
    -P(x,y,v1) | -P(x,z,v2) | -S(y,z,v3) | -P(x,v3,v4) | S(v1,v2,v4).
    -P(x,y,v1) | -P(x,z,v2) | -S(y,z,v3) | -S(v1,v2,v4) | P(x,v3,v4).
    -P(y,x,v1) | -P(z,x,v2) | -S(y,z,v3) | -P(v3,x,v4) | S(v1,v2,v4).
    -P(y,x,v1) | -P(z,x,v2) | -S(y,z,v3) | -S(v1,v2,v4) | P(v3,x,v4).
  
    % multiplication is well-defined
    -P(x,y,u) | -P(x,y,v) | eq(u,v).
  
    -eq(u,v) | -P(u,x,y) | P(v,x,y).
    -eq(u,v) | -P(x,u,y) | P(x,v,y).
    -eq(u,v) | -P(x,y,u) | P(x,y,v).
    -eq(u,v) | eq(f(u,x),f(v,x)).
    -eq(u,v) | eq(f(x,u),f(x,v)).
  
    % the square of every x is x
    P(x,x,x).

    % Denial:  there exist elements that don't commute
    P(a,b,c).
    -P(b,a,c).


Theorem 4:
  
    % condensed detachment
    -P(x) | -P(e(x,y)) | P(y).
  
    % the formula XGK
    P(e(x,e(e(y,e(z,x)),e(z,y)))).

    % the negation of PYO, which is a known single axiom
    -P(e(e(e(a,e(b,c)),c),e(b,a))).
  

Theorem 5:
    % condensed detachemt
    -P(x) | -P(i(x,y)) | P(y).
    
    % formula suspected of being a single axiom    
    P(i(i(i(x,y),z),i(i(z,x),i(u,x)))).

    % the negation of a known single axiom
    -P(i(i(a,b),i(i(b,c),i(a,c)))).


Theorem 6:
    % condensed detachemt
    -P(x) | -P(i(x,y)) | P(y).

    P(i(x,i(y,x))).

    P(i(i(x,y),i(i(y,z),i(x,z)))).

    P(i(i(i(x,y),y),i(i(y,x),x))).

    P(i(i(n(x),n(y)),i(y,x))).

    -P(i(i(a,b),i(i(c,a),i(c,b)))).


Theorem 7:
    % condensed detachemt
    -P(x) | -P(i(x,y)) | P(y).

    P(i(x,i(y,x))).

    P(i(i(x,y),i(i(y,z),i(x,z)))).

    P(i(i(i(x,y),y),i(i(y,x),x))).

    P(i(i(n(x),n(y)),i(y,x))).

    -P(i(i(a,b),i(n(b),n(a)))).
\end{verbatim}
  
\subsection*{Rules:}

Systems must accept as input only the clauses (in whatever format is most
convenient); no parameters, designation of set-of-support, or weighting
templates are allowed.

Systems must be run during CADE-11, June 1992, either at the site of
the conference or over networks.  They may be run on any machines.

Systems may not be ``over specialized''; for example, a system that
simply printed the proofs of the above problems would be unacceptable.
A system that had builtin mechanisms for weighting just the terms that
were known to appear in these proofs would also be unacceptable.  I
find it difficult to make this notion precise, so I will simply say
that the competitors can vote on any serious question that comes up
(and I will trust to their basic decency and goodwill in settling any
disagreements).

Grading of systems will be done as follows:

\begin{enumerate}
\item All systems will be run, and the times for deriving the seven
	   proofs will be noted.  A system must prove  all problems 
           to win, unless no system successfully proves them all.

\item A system will receive points for placing in the top three for each
	   of the problems -- three points for 1st, two points for 2nd, and
           one point for 3rd.

\item The system with the highest score wins the prize.
\end{enumerate}

I offer this prize, since there are many exciting opportunities offered by
current developments in massively-parallel machines, compilation technology,
and indexing technology.  There are well-known advocates of a variety of
alternative approaches.  I encourage everyone to compete in this fairly
circumscribed context in a friendly, common exploration.  There will probably
be at least two competing systems formed by my friends at Argonne (who hold
somewhat differing views on detailed issues).  I will attempt to build a
system around the use of FDB (a set of C routines that implement the basic
operations), Strand (a committed-choice logic programming language), and
distributed memory multiprocessors.  I would prefer to implement this approach
in cooperation with a group at some university.  In any event, I would happily
share my current views, code, and outlook with anyone that wished to build a
competing system.  Access to high-performance hardware will clearly be useful,
and I will attempt to make sure that any team that develops a system gets
access to large SIMD or MIMD machines.


\section*{Equality-based Deduction}

There are a number of good equality-based systems that have emerged over the
last five years.  This competition will be based on the following set of
problems:

\subsection*{Theorems:}

{\bf Problem} 1: In a group, if $x * x * x = e$ for all $x$ in the group, then
the commutator $h(h(x, y), y) = e$ for all $x$ and $y$, where $h(x, y)$ is defined
as $x * y * g(x) * g(y).$

{\bf Problem} 2: In a Robbins algebra, if there exists $c$, such that
$o(c, c) = c$, then $o(x, n(o(x, n(x)))) = x$ for all $x$.

{\bf Problem} 3: In a ternary Boolean algebra with the third axiom removed,
it is true that $f(x, g(x), y) = y$ for all $x$ and $y$.

{\bf Problem} 4: The group theory specified by the axiom 
x * i((i(i(y) * (i(x) * w)) * z) * i(y * z)) = w 
implies the associativity of *.

{\bf Problem} 5: The Wajsberg algebra formulation of a conjecture by
Lukasiewicz.
    
{\bf Problem} 6: Find a fixed-point combinator in $\{B,W,M\}$.
           
{\bf Problem} 7:  A ring in which $x * x * x = x$ is commutative.

{\bf Problem} 8: Find a fixed-point combinator in $\{B,W\}$.
          
{\bf Problem} 9.
	Show that the three Moufang identities imply the
	skew-symmetry of $f(w,x,y,z) = (w*x,y,z) - x*(w,y,z) - (x,y,z)*w.$

{\bf Problem} 10.  Show in a right alternative ring (just the right
alternative axiom) that $(((x,x,y)*x)*(x,x,y)) = 0$.

\subsection*{Axioms for each problem:}

\begin{verbatim}
Axioms for Theorem 1: 

1  f(e, x) = x
2  f(g(x), x) = e
3  f(f(x, y), z) = f(x, f(y, z))
4  h(x, y) = f(x, f(y, f(g(x), g(y))))

5  f(x, f(x, x)) = e
6  h(h(a, b), b) != e
\end{verbatim}

\begin{verbatim}
Axioms for Theorem 2:

1  o(x, y) = o(y, x)
2  o(o(x, y), z) = o(x, o(y, z))
3  n(o(n(o(x, y)), n(o(x, n(y))))) = x

4  o(c, c) = c
5  o(a, n(o(a, n(a)))) != a
\end{verbatim}

\begin{verbatim}
Axioms for Theorem 3:

1  f(f(v, w, x), y, f(v, w, z)) = f(v, w, f(x, y, z))
2  f(y, x, x) = x
4  f(x, x, y) = x
5  f(g(y), y, x) = x

6  f(a, g(a), b) != b
\end{verbatim}

\begin{verbatim}
Axioms for Theorem 4:

1  f(x, i(f(f(i(f(i(y), f(i(x), w))), z), i(f(y, z))))) = w

2  f(a, f(b, c)) != f(f(a, b), c)
\end{verbatim}

\begin{verbatim}       
Axioms for Theorem 5:

1 i(T,x) = x
2 i(i(x,y),i(i(y,z),i(x,z))) = T
3 i(i(x,y),y) = i(i(y,x),x)
4 i(i(n(x),n(y)),i(y,x)) = T

5 i(i(i(a,b),i(b,a)),i(b,a)) != T
\end{verbatim}

\begin{verbatim}
Axioms for Theorem 6:

1 a(a(a(B,x),y),z) = a(x,a(y,z))
2 a(a(W,x),y) = a(a(x,y),y)
3 a(M,x) = a(x,x)

4 a(y,f(y)) != a(f(y),a(y,f(y)))
\end{verbatim}

\begin{verbatim}
Axioms for Theorem 7:

1  j(x, y) = j(y, x)
2  j(j(x, y), z) = j(x, j(y, z))
3  j(x, 0) = x
4  j(x, i(x)) = 0
5  f(x, j(y, z)) = j(f(x, y), f(x, z))
6  f(j(y, z), x) = j(f(y, x), f(z, x))
7  f(f(x, y), z) = f(x, f(y, z))

8  f(x, f(x,x)) = x
9  f(a,b) != f(b,a)
\end{verbatim}
 
\begin{verbatim} 
Axioms for Theorem 8:

1 a(a(a(B,x),y),z) = a(x,a(y,z))
2 a(a(W,x),y) = a(a(x,y),y)

3 a(y,f(y)) != a(f(y),a(y,f(y)))
\end{verbatim}

\begin{verbatim}
Common Axioms for Theorems 9 and 10.

1 j(x,y) = j(y,x)                     * Commutativity for addition 
2 j(x,j(y,z)) = j(j(x,y),z)           * Associativity for addition 
3 j(x,0) = x                          * Additive identity 
4 j(0,x) = x
5 j(x,g(x)) = 0                       * Additive inverse 
6 j(g(x),x) = 0
7 g(0) = 0                             * Inverse of zero
8 j(x,j(g(x),y)) = y
9 g(j(x,y)) = j(g(y),g(x))
10 g(g(x)) = x                          
11 f(x,0) = 0
12 f(0,x) = 0
13 f(g(x),g(y)) = f(x,y)
14 g(f(x,y)) = f(g(x),y)
15 g(f(x,y)) = f(x,g(y))
16 f(x,j(y,z)) = j(f(x,y),f(x,z))     * Distributivity of addition 
17 f(j(x,y),z) = j(f(x,z),f(y,z))     * Distributivity of addition 
18 f(f(x,y),y) = f(x,f(y,y))          * Right alternative law 
19 f(f(x,x),y) = f(x,f(x,y))          * Left alternative law 
20 a(x,y,z) = j(f(f(x,y),z),g(f(x,f(y,z))))       * Associator
21 c(x,y) = j(f(y,x)),g(f(x,y))                  * Commutator 
22 f(w,x,y,z) = j(j(a(f(w,x),y,z),g(f(x,a(w,y,z)))),g(f(a(x,y,z),w)))  
   * Definition of f
23 f(f(a(x,x,y),x),a(x,x,y)) = 0                       
   * Middle associator identity 
\end{verbatim}

The following clauses are needed for proving problem 9.

\begin{verbatim}
24 f(x,f(y,f(z,y)) = f(c(f(x,y),z),y))                 * Right Moufang
25 f(f(y,f(z,y)),x) = f(y,c(z,f(y,x)))                 * Left Moufang
26 f(f(x,y),f(z,x)) = f(f(x,f(y,z)),x)                 * Middle Moufang

27a f(x,y,z,w) != g(f(y,x,z,w))                        * Denial of 9.
\end{verbatim} 

For problem 10 you should use the same axioms as above excepting 19,
22 and 27a with the addition of 27b.

\begin{verbatim}
27b	  f(f(a(a,a,b),b),a(a,a,b))) != 0 	       * Denial of 10.
\end{verbatim}

Note that $(x,y,z)$ is an abbreviation for the associator $a(x,y,z)$
defined above.

\subsection*{Rules:}

The basic rules for this competition are the same as in the previous
competition, except that programs are allowed to accept as input a term
ordering for each of the problems.  


\end{document}