\chapter{Invoking {\em akb} on a single axiom of free groups}
\normalsize
\rm
This example is another equational axiomatization of free groups
using a single axiom and the right division
operator /. Higman showed that this single axiom is
sufficient to characterize free groups. This example illustrates
the use of the automatic completion command in RRL in which
no user assistance is required; the program itself determines
in a systematic (and exhaustive) manner the precedence
relation over the function symbols to orient equations. RRL
provides facilities to change parameters to suit this automatic
facility to families of examples. This examples also illustrates
the mechanism in RRL to introduce new function symbols when
equations are generated in which neither side has variables
which include the variables of the other side. The goal here is
also to generate a complete system, which is closely related
to the complete system in the previous Appendix. RRL can also
be used to show equivalence of two different axiomatizations
of free groups.

For generating a complete system from an axiomatization in the automatic
mode of orienting rules from equations, the user simply invokes
{\em akb} on the axiomatization. Of course, 
the user can provide hints to the termination algorithm
by specifying precedence on some function symbols using
the {\em oper prec} command before invoking {\em akb}. Further,
the parameters to the termination algorithm
can be suitably chosen using the {\em option autoorder} command
(see Section 3.5.2). (In the transcript below, minimal tracing
information is being printed.)

\small
\tt
\begin{verbatim}
Type Add, Akb, Auto, Break, Clean, Delete, Grammar, History, Init, Kb, List,
     Load, Log, Makerule, Narrow, Norm, Option, Operator, Prove, Quit, Read,
     Refute, Save, Statics, Suffice, Undo, Unlog, Write or Help.
\end{verbatim}
RRL-> {\em read group1}
\begin{verbatim}
Equations read in are:
 1. (x / ((((x / x) / y) / z) / (((x / x) / x) / z))) == y  [user, 1]
Time = 0.13 sec
Type Add, Akb, Auto, Break, Clean, Delete, Grammar, History, Init, Kb, List,
     Load, Log, Makerule, Narrow, Norm, Option, Operator, Prove, Quit, Read,
     Refute, Save, Statics, Suffice, Undo, Unlog, Write or Help.
\end{verbatim}
RRL-> {\em option trace 1}
\begin{verbatim}
Time = 0.02 sec
Type Add, Akb, Auto, Break, Clean, Delete, Grammar, History, Init, Kb, List,
     Load, Log, Makerule, Narrow, Norm, Option, Operator, Prove, Quit, Read,
     Refute, Save, Statics, Suffice, Undo, Unlog, Write or Help.
\end{verbatim}
RRL-> {\em akb}
\begin{verbatim}
----- Step 1 -----
This eqn cannot be oriented into a rule: 
  (x / ((y / z) / (((x / x) / x) / z))) == (((((x / x) / (x / x)) / y) / z1) / 
                          ((((x / x) / (x / x)) / (x / x)) / z1))  [1, 1]
Introduce new operator.
----- Step 2 -----
Trying to orient equation: 
  (x / ((y / z) / (((x / x) / x) / z))) == f1(x, y)  [1, 1]
  To prove:  (x / ((y / z) / (((x / x) / x) / z)))  >  f1(x, y)
  Here are some precedence suggestions:
       1.  / > f1
Either type a list of numbers or
Type Abort, Display, Drop, Equiv, LR, MakeEq, Operator, Postpone, Quit, RL,
     Status, Superpose, Twoway, Undo or Help.
RRL>KB> auto
Precedence relation, / = f1, is added.
----- Step 3 -----
Trying to orient equation: 
  f1(x, f1((x / x), y)) == ((y / z) / ((((x / x) / (x / x)) / (x / x)) / z))  [3, 4]
  To prove:  ((y / z) / ((((x / x) / (x / x)) / (x / x)) / z))  >  f1(x, f1((x / x), y))
I have no precedence suggestions.  
Try doing Equiv or Status.
Type Abort, Display, Drop, Equiv, LR, MakeEq, Operator, Postpone, Quit, RL,
     Status, Superpose, Twoway, Undo or Help.
RRL>KB> auto
The equation is postponed.
----- Step 4 -----
This eqn cannot be oriented into a rule: 
  f1(x, (x / x)) == (y / y)  [16, 4]
Introduce new operator.
----- Step 5 -----
Trying to orient equation: 
  f1(x, (x / x)) == f2  [16, 4]
  To prove:  f1(x, (x / x))  >  f2
  Here are some precedence suggestions:
       1.  f1 > f2
Either type a list of numbers or
Type Abort, Display, Drop, Equiv, LR, MakeEq, Operator, Postpone, Quit, RL,
     Status, Superpose, Twoway, Undo or Help.
RRL>KB> auto
Precedence relation, f1 = f2, is added.
----- Step 6 -----
This eqn cannot be oriented into a rule: 
  (((f2 / y) / z) / (f2 / z)) == f1(x, y)  [deleted, 18]
Introduce new operator.
----- Step 7 -----
Trying to orient equation: 
  (((f2 / y) / z) / (f2 / z)) == f3(y)  [deleted, 18]
  To prove:  (((f2 / y) / z) / (f2 / z))  >  f3(y)
  Here are some precedence suggestions:
       1.  / > f3
Either type a list of numbers or
Type Abort, Display, Drop, Equiv, LR, MakeEq, Operator, Postpone, Quit, RL,
     Status, Superpose, Twoway, Undo or Help.
RRL>KB> auto
Precedence relation, / = f3, is added.
----- Step 8 -----
Trying to orient equation: 
  (x / (y / f3(y1))) == ((x / y1) / y)  [52, 57]
  To prove:  (x / (y / f3(y1)))  >  ((x / y1) / y)
        or:  (x / (y / f3(y1)))  <  ((x / y1) / y)
I have no precedence suggestions.  
Try doing Equiv or Status.
Type Abort, Display, Drop, Equiv, LR, MakeEq, Operator, Postpone, Quit, RL,
     Status, Superpose, Twoway, Undo or Help.
RRL>KB> auto
The equation is postponed.
----- Step 9 -----
Trying to orient equation: 
  (x / (y / f3(y1))) == ((x / y1) / y)  [63, 57]
  To prove:  (x / (y / f3(y1)))  >  ((x / y1) / y)
        or:  (x / (y / f3(y1)))  <  ((x / y1) / y)
I have no precedence suggestions.  
Try doing Equiv or Status.
Type Abort, Display, Drop, Equiv, LR, MakeEq, Operator, Postpone, Quit, RL,
     Status, Superpose, Twoway, Undo or Help.
RRL>KB> auto
The equation is postponed.
----- Step 10 -----
Trying to orient equation: 
  (x / (y / (f3(z) / y1))) == (((x / z) / y1) / y)  [57, 57]
  To prove:  (x / (y / (f3(z) / y1)))  >  (((x / z) / y1) / y)
        or:  (x / (y / (f3(z) / y1)))  <  (((x / z) / y1) / y)
I have no precedence suggestions.  
Try doing Equiv or Status.
Type Abort, Display, Drop, Equiv, LR, MakeEq, Operator, Postpone, Quit, RL,
     Status, Superpose, Twoway, Undo or Help.
RRL>KB> auto
The equation is postponed.
----- Step 11 -----
Trying to orient equation: 
  ((y / f3(z)) / f3(z1)) == (y / (f3(z1) / z))  [74, 50]
  To prove:  ((y / f3(z)) / f3(z1))  >  (y / (f3(z1) / z))
        or:  ((y / f3(z)) / f3(z1))  <  (y / (f3(z1) / z))
I have no precedence suggestions.  
Try doing Equiv or Status.
Type Abort, Display, Drop, Equiv, LR, MakeEq, Operator, Postpone, Quit, RL,
     Status, Superpose, Twoway, Undo or Help.
RRL>KB> auto
The equation is postponed.
----- Step 12 -----
Trying to orient equation: 
  ((y / f3(z)) / y1) == (y / (y1 / z))  [74, 51]
  To prove:  ((y / f3(z)) / y1)  >  (y / (y1 / z))
        or:  ((y / f3(z)) / y1)  <  (y / (y1 / z))
I have no precedence suggestions.  
Try doing Equiv or Status.
Type Abort, Display, Drop, Equiv, LR, MakeEq, Operator, Postpone, Quit, RL,
     Status, Superpose, Twoway, Undo or Help.
RRL>KB> auto
The equation is postponed.
----- Step 13 -----
Trying to orient equation: 
  ((y / f3(z)) / f3(z1)) == (y / (f3(z1) / z))  [74, 50]
  To prove:  ((y / f3(z)) / f3(z1))  >  (y / (f3(z1) / z))
        or:  ((y / f3(z)) / f3(z1))  <  (y / (f3(z1) / z))
I have no precedence suggestions.  
Try doing Equiv or Status.
Type Abort, Display, Drop, Equiv, LR, MakeEq, Operator, Postpone, Quit, RL,
     Status, Superpose, Twoway, Undo or Help.
RRL>KB> auto
Operator, /, given status: rl
----- Step 14 -----
Your system is canonical.
 [20] (y / y) ---> f2  [16, 4]
 [23] f1(x, y) ---> f3(y)  [deleted, 18]
 [24] f3(f2) ---> f2  [deleted, 21]
 [26] (x / f2) ---> x  [deleted, 16]
 [37] (f2 / y) ---> f3(y)  [26, 22]
 [39] f3(f3(y)) ---> y  [deleted, 35]
 [50] ((y / z) / f3(z)) ---> y  [39, 49]
 [51] ((y / f3(y1)) / y1) ---> y  [39, 50]
 [64] f3((y / x)) ---> (x / y)  [50, 52]
 [86] (y / (y1 / z)) ---> ((y / f3(z)) / y1)  [74, 51]
Processor time used                = 18.5 sec
Number of rules generated          = 86
Number of critical pairs           = 127
Time spent in normalization        = 9.42 sec          (50.90 percent of time)
Time spent while adding rules      = 5.03 sec          (27.21 percent of time)
  (keeping rule set reduced)
Time spent in unification          = 1.12 sec          (6.04 percent of time)
Time spent in ordering             = 3.55 sec          (19.19 percent of time)
Total processor time used (include 'undo' action) = 23.07 sec
Time = 23.3 sec
Type Add, Akb, Auto, Break, Clean, Delete, Grammar, History, Init, Kb, List,
     Load, Log, Makerule, Narrow, Norm, Option, Operator, Prove, Quit, Read,
     Refute, Save, Statics, Suffice, Undo, Unlog, Write or Help.
\end{verbatim}
RRL-> {\em a}\\
Type your equations, enter a ']' when done.\\
{\em x * y == x / f3(y)\\
i(x) == f3(x)\\
e == f2\\
]}
\begin{verbatim}
Equations read in are:
 1. (x * y) == (x / f3(y))  [user, 11]
 2. i(x) == f3(x)  [user, 12]
 3. e == f2  [user, 13]
New constant set is: { e }
Time = 0.03333333333333854 sec
Type Add, Akb, Auto, Break, Clean, Delete, Grammar, History, Init, Kb, List,
     Load, Log, Makerule, Narrow, Norm, Option, Operator, Prove, Quit, Read,
     Refute, Save, Statics, Suffice, Undo, Unlog, Write or Help.
\end{verbatim}
RRL-> {\em oper prec}
\begin{verbatim}
Type operators in decreasing precedence.
  (eg. 'i * e' to set i > * > e) 
\end{verbatim}
--> {\em i * e /}
\begin{verbatim}
Precedence relation, i > *, is added.
Precedence relation, * > e, is added.
Precedence relation, e > /, is added.
Time = 0.01666666666666572 sec
Type Add, Akb, Auto, Break, Clean, Delete, Grammar, History, Init, Kb, List,
     Load, Log, Makerule, Narrow, Norm, Option, Operator, Prove, Quit, Read,
     Refute, Save, Statics, Suffice, Undo, Unlog, Write or Help.
\end{verbatim}
RRL-> {\em opt tr 2}
\begin{verbatim}
Time = 0.01666666666666572 sec
Type Add, Akb, Auto, Break, Clean, Delete, Grammar, History, Init, Kb, List,
     Load, Log, Makerule, Narrow, Norm, Option, Operator, Prove, Quit, Read,
     Refute, Save, Statics, Suffice, Undo, Unlog, Write or Help.
\end{verbatim}
RRL-> {\em kb}
\begin{verbatim}
----- Step 15 -----
Adding rule:  [92] (x * y) ---> (x / f3(y))  [user, 11]
Adding rule:  [93] i(x) ---> f3(x)  [user, 12]
Adding rule:  [94] e ---> f2  [user, 13]
Computing critical pairs with: 
   [94] e ---> f2  [user, 13]
Computing critical pairs with: 
   [93] i(x) ---> f3(x)  [user, 12]
Computing critical pairs with: 
   [92] (x * y) ---> (x / f3(y))  [user, 11]

Your system is canonical.

 [22] (y / y) ---> f2  [17, 4]
 [25] f1(x, y) ---> f3(y)  [deleted, 19]
 [26] f3(f2) ---> f2  [deleted, 23]
 [28] (x / f2) ---> x  [deleted, 17]
 [40] (f2 / y) ---> f3(y)  [28, 24]
 [42] f3(f3(y)) ---> y  [deleted, 38]
 [54] ((y / z) / f3(z)) ---> y  [42, 53]
 [56] ((y / f3(y1)) / y1) ---> y  [42, 54]
 [69] f3((y / x)) ---> (x / y)  [54, 57]
 [91] (y / (y1 / z)) ---> ((y / f3(z)) / y1)  [79, 56]
 [92] (x * y) ---> (x / f3(y))  [user, 11]
 [93] i(x) ---> f3(x)  [user, 12]
 [94] e ---> f2  [user, 13]
  ... ....

Type Add, Akb, Auto, Break, Clean, Delete, Grammar, History, Init, Kb, List,
     Load, Log, Makerule, Narrow, Norm, Option, Operator, Prove, Quit, Read,
     Refute, Save, Statics, Suffice, Undo, Unlog, Write or Help.
\end{verbatim}
RRL-> {\em prove}
\begin{verbatim}
Type equation to prove in the format:  L == R (if C) 
Enter a ']' to exit when no equation is given.
\end{verbatim}
{\em e * x == x}
\begin{verbatim}
Yes, it is equational theorem.
Time = 0.03333333333333144 sec
Type Add, Akb, Auto, Break, Clean, Delete, Grammar, History, Init, Kb, List,
     Load, Log, Makerule, Narrow, Norm, Option, Operator, Prove, Quit, Read,
     Refute, Save, Statics, Suffice, Undo, Unlog, Write or Help.
\end{verbatim}
RRL-> {\em prove}
\begin{verbatim}
Type equation to prove in the format:  L == R (if C) 
Enter a ']' to exit when no equation is given.
\end{verbatim}
{\em i(x) * x == e}
\begin{verbatim}
Yes, it is equational theorem.
Time = 0.0 sec
Type Add, Akb, Auto, Break, Clean, Delete, Grammar, History, Init, Kb, List,
     Load, Log, Makerule, Narrow, Norm, Option, Operator, Prove, Quit, Read,
     Refute, Save, Statics, Suffice, Undo, Unlog, Write or Help.
\end{verbatim}
RRL-> {\em prove}
\begin{verbatim}
Type equation to prove in the format:  L == R (if C) 
Enter a ']' to exit when no equation is given.
\end{verbatim}
{\em (x * y) * z == x * (y * z)}
\begin{verbatim}
Yes, it is equational theorem.
Time = 0.06666666666666288 sec
\end{verbatim}
\rm
\normalsize
