\chapter{Illustration of the clausal superposition method}
\normalsize
\rm
This example illustrates a refutational proof method based on
conditional rewrite rules. Suppose the file {\em geom.eqn} contains 
the following clauses:
\begin{quote}
\small
\begin{verbatim}
; For any two distinct points, there is a unique line through them.
on(z1, ln(z1, z2)) or (z1 = z2) or not point(z1) or not point(z2)
on(z2, ln(z1, z2)) or (z1 = z2) or not point(z1) or not point(z2)
line(ln(z1, z2)) or (z1 = z2) or not point(z1) or not point(z2)
not on(z1, y3) or (z1 = z2) or not on(z2, y3) or (y3 = y4) or not on(z1, y4) or  
  not on(z2, y4) or not point(z1) or not point(z2) or not line(y3) or not line(y4)
; For any line, there are at least two distinct points on the line.
on(pt1(y1), y1) or not line(y1) 
on(pt2(y1), y1) or not line(y1) 
point(pt1(y1)) or not line(y1) 
point(pt2(y1)) or not line(y1) 
not (pt1(y1) = pt2(y1)) or not line(y1)
; For any line, there is a point not on the line.
not on(pt3(y1), y1) or not line(y1)
point(pt3(y1)) or not line(y1) 
; There is a point on every line.
on(pt, y1) or not line(y1)
; There exists a line.
line(a)
; There is a point on every line.
point(pt)       
\end{verbatim}
\end{quote}
\RRL proves that the above set is inconsistent by generating the rule
{\tt false ---> true}.
\scriptsize
\tt
\begin{verbatim}
Type Add, Akb, Auto, Break, Clean, Delete, Grammar, History, Init, Kb, List,
     Load, Log, Makerule, Narrow, Norm, Option, Operator, Prove, Quit, Read,
     Refute, Save, Statics, Suffice, Undo, Unlog, Write or Help.
RRL-> option trace 1

Type Add, Akb, Auto, Break, Clean, Delete, Grammar, History, Init, Kb, List,
     Load, Log, Makerule, Narrow, Norm, Option, Operator, Prove, Quit, Read,
     Refute, Save, Statics, Suffice, Undo, Unlog, Write or Help.
RRL-> option prove 
Following methods are available:
  Deduction:
     (f) forward reasoning --- using completion
     (s) refutation        --- using boolean-ring method
     (c) refutation        --- using conditional rules
  Induction:
     (e) explicit          --- using the cover-set method
     (s) inductionless     --- using the test set method
     (g) inductionless     --- using the ground-reducibility method
Please make your choice: 
(current value: s) (f,c,e,s,g) c

Type Add, Akb, Auto, Break, Clean, Delete, Grammar, History, Init, Kb, List,
     Load, Log, Makerule, Narrow, Norm, Option, Operator, Prove, Quit, Read,
     Refute, Save, Statics, Suffice, Undo, Unlog, Write or Help.
RRL-> read geom
New constant set is: { a, pt }
Equations read in are:
 1. on(z1, ln(z1, z2)) == true  if  not((z1 = z2)) & (point(z1) and point(z2))  [user, 1]
 2. on(z2, ln(z1, z2)) == true  if  not((z1 = z2)) & (point(z1) and point(z2))  [user, 2]
 3. line(ln(z1, z2)) == true  if  not((z1 = z2)) & (point(z1) and point(z2))  [user, 3]
 4. on(z1, y3) == false  if  not((z1 = z2)) & (on(z2, y3) and (not((y3 = y4)) and (on(z1, y4) and 
      (on(z2, y4) and (point(z1) and (point(z2) and (line(y3) and line(y4))))))))  [user, 4]
 5. on(pt1(y1), y1) == true  if  line(y1)  [user, 5]
 6. on(pt2(y1), y1) == true  if  line(y1)  [user, 6]
 7. point(pt1(y1)) == true  if  line(y1)  [user, 7]
 8. point(pt2(y1)) == true  if  line(y1)  [user, 8]
 9. (pt1(y1) = pt2(y1)) == false  if  line(y1)  [user, 9]
10. on(pt3(y1), y1) == false  if  line(y1)  [user, 10]
11. point(pt3(y1)) == true  if  line(y1)  [user, 11]
12. on(pt, y1) == true  if  line(y1)  [user, 12]
13. line(a)  [user, 13]
14. point(pt)  [user, 14]

Type Add, Akb, Auto, Break, Clean, Delete, Grammar, History, Init, Kb, List,
     Load, Log, Makerule, Narrow, Norm, Option, Operator, Prove, Quit, Read,
     Refute, Save, Statics, Suffice, Undo, Unlog, Write or Help.
RRL-> kb

This eqn cannot be oriented into a rule: 
  pt2(y3) == z2  if  
        { line(y4),
          point(z2),
          on(z2, y4),
          on(pt2(y3), y4),
          (y4 = y3) <=> false,
          on(z2, y3),
          line(y3) }   [6, 4]
Type left_right, makeeq, operator, postpone, right_left, undo, quit, help --> p

This eqn cannot be oriented into a rule: 
  pt1(y3) == z2  if  
        { line(y4),
          point(z2),
          on(z2, y4),
          on(pt1(y3), y4),
          (y4 = y3) <=> false,
          on(z2, y3),
          line(y3) }   [5, 4]
Type left_right, makeeq, operator, postpone, right_left, undo, quit, help --> p

This eqn cannot be oriented into a rule: 
  ln(z1, z2) == y4  if  
        { (pt = z2) <=> false,
          on(z2, y4),
          line(y4),
          point(z2),
          point(z1),
          (z2 = z1) <=> false }   [2, 15]
Type left_right, makeeq, operator, postpone, right_left, undo, quit, help --> p

This eqn cannot be oriented into a rule: 
  ln(z2, z21) == y4  if  
        { (pt = z2) <=> false,
          on(z2, y4),
          line(y4),
          point(z21),
          point(z2),
          (z2 = z21) <=> false }   [1, 15]
Type left_right, makeeq, operator, postpone, right_left, undo, quit, help --> p

In on(z2, y3) == true  if  
        { point(z2),
          (pt2(y3) = z2) <=> false,
          line(y3),
          (pt2(y3) = pt) <=> false }   [23, 2]
    the premise (pt2(y3) = z2) <=> false is released,
    because the equation is true under its negation.

In on(z2, y3) == true  if  
        { point(z2),
          (pt1(y3) = z2) <=> false,
          line(y3),
          (pt1(y3) = pt) <=> false }   [21, 2]
    the premise (pt1(y3) = z2) <=> false is released,
    because the equation is true under its negation.

Inconsistent Relation Derived:
     [32] false ---> true  [deleted, 13]

Input #1, produces:
      [1] on(z1, ln(z1, z2)) ---> true if 
        { (z2 = z1) <=> false,
          point(z1),
          point(z2) } 
Input #2, produces:
      [2] on(z2, ln(z1, z2)) ---> true if 
        { (z2 = z1) <=> false,
          point(z1),
          point(z2) } 
Input #3, produces:
      [3] line(ln(z1, z2)) ---> true if 
        { (z2 = z1) <=> false,
          point(z1),
          point(z2) } 
Input #4, produces:
      [4] on(z1, y3) ---> false if 
        { (z2 = z1) <=> false,
          on(z2, y3),
          (y4 = y3) <=> false,
          on(z1, y4),
          on(z2, y4),
          point(z1),
          point(z2),
          line(y3),
          line(y4) } 
Input #5, produces:
      [5] on(pt1(y1), y1) ---> true if  { line(y1) } 
Input #6, produces:
      [6] on(pt2(y1), y1) ---> true if  { line(y1) } 
Input #7, produces:
      [7] point(pt1(y1)) ---> true if  { line(y1) } 
Input #8, produces:
      [8] point(pt2(y1)) ---> true if  { line(y1) } 
Input #9, produces:
      [9] (pt2(y1) = pt1(y1)) ---> false if  { line(y1) } 
Input #10, produces:
     [10] on(pt3(y1), y1) ---> false if  { line(y1) } 
Input #11, produces:
     [11] point(pt3(y1)) ---> true if  { line(y1) } 
Input #12, produces:
     [12] on(pt, y1) ---> true if  { line(y1) } 
Input #13, produces:
     [13] line(a) ---> true
Input #14, produces:
     [14] point(pt) ---> true
Rule [12] superposed into Rule [4], reduced by Rules [12], and [14], produces:
     [15] on(z2, y3) ---> false if 
        { line(y4),
          point(z2),
          on(z2, y4),
          (y4 = y3) <=> false,
          (pt = z2) <=> false,
          line(y3) } 
Rule [6] superposed into Rule [15], reduced by Rules [8], [14], and [12], produces:
     [18] on(pt2(y3), y4) ---> false if 
        { line(y3),
          (pt2(y3) = pt) <=> false,
          (y4 = y3) <=> false,
          line(y4) } 
Rule [5] superposed into Rule [15], reduced by Rule [7], produces:
     [19] on(pt1(y3), y4) ---> false if 
        { line(y3),
          (pt1(y3) = pt) <=> false,
          (y4 = y3) <=> false,
          line(y4) } 
Rule [1] superposed into Rule [19], reduced by Rules [3], and [7], produces:
     [21] ln(pt1(y3), z2) ---> y3 if 
        { (pt1(y3) = pt) <=> false,
          line(y3),
          point(z2),
          (pt1(y3) = z2) <=> false } 
Rule [1] superposed into Rule [18], reduced by Rules [3], and [8], produces:
     [23] ln(pt2(y3), z2) ---> y3 if 
        { (pt2(y3) = pt) <=> false,
          line(y3),
          point(z2),
          (pt2(y3) = z2) <=> false } 
Rule [23] superposed into Rule [2], reduced by Rules [6], and [8], produces:
     [24] on(z2, y3) ---> true if 
        { point(z2),
          line(y3),
          (pt2(y3) = pt) <=> false } 
Rule [10] superposed into Rule [24], reduced by Rules [11], [8], and [6], produces:
     [25] line(y3) ---> false if  { (pt2(y3) = pt) <=> false } 
Rule [13] superposed into Rule [25], reduced by Rules [24], and [25], produces:
     [26] pt2(a) ---> pt
Rule [26] superposed into Rule [9], reduced by Rule [13], produces:
     [28] (pt1(a) = pt) ---> false
Rule [21] superposed into Rule [2], reduced by Rules [5], and [7], produces:
     [30] on(z2, y3) ---> true if 
        { point(z2),
          line(y3),
          (pt1(y3) = pt) <=> false } 
Rule [10] superposed into Rule [30], reduced by Rules [11], and [25], produces:
     [31] line(y3) ---> false if  { (pt1(y3) = pt) <=> false } 
Rule [13], deleted by Rule [31], reduced by Rules [28], and [31], produces:
     [32] false ---> true

The proof length (unifications) is 12.
Processor time used                  = 24.17 sec
\end{verbatim}
\rm
\normalsize
