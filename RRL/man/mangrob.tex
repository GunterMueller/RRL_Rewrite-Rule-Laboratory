\chapter {Illustration of the Gr\"{o}bner basis method} 
\section{An example from Propositional Calculus}
\normalsize 
\rm 
An example from propositional calculus is illustrated, which
is a modification of an
example from \cite{Carroll}. We are given a set of hypotheses
describing a situation, and the goal is to deduce everything about the
situation.  \RRL can be used to generate a complete system for these
hypotheses which can be used to answer queries about the situation.
\begin{center}
\bf Definition of Symbols
\end{center}
\begin{tabular}{clcl}
a & some English sing & c & some monitors are awake \\
e & some Scotch dance & k & some Irish fight \\
m & some Welsh eat & r & some Germans play \\
t & some Eleven are not oiling & &
\end{tabular}
%\vspace{0.5in}
\small
\tt
\begin{verbatim}
Type Add, Akb, Auto, Break, Clean, Delete, Dump, Grammar, Init, Kb, List,
     Log, Makerule, Narrow, Norm, Order, Option, Operator, Prove, Quit, Read,
     Refute, Stats, Suffic, Undo, Unlog, Write or Help.
\end{verbatim}
RRL-> {\em r carroll2.pc}
\begin{verbatim}
Equations read in are:
 1. a  [user, 1]
 2. ((e and k) -> m)  [user, 2]
 3. (r -> t)  [user, 3]
 4. (r -> (e or not(m)))  [user, 4]
 5. ((not(e) and not(k)) -> r)  [user, 5]
 6. ((c and (m and e)) -> false)  [user, 6]
 7. ((not(r) and (not(m) and k)) -> false)  [user, 7]
 8. ((not(e) and r) -> not(a))  [user, 8]
 9. ((a and not(c)) -> not(k))  [user, 9]
10. ((c and t) -> e)  [user, 10]
New constant set is: { c, t, r, m, k, e, a }
Time = 0.2000000000000028 sec
Type Add, Akb, Auto, Break, Clean, Delete, Dump, Grammar, Init, Kb, List,
     Log, Makerule, Narrow, Norm, Order, Option, Operator, Prove, Quit, Read,
     Refute, Stats, Suffic, Undo, Unlog, Write or Help.
\end{verbatim}
RRL-> {\em kb}
\begin{verbatim}
----- Step 1 -----
Adding rule:   [1] a ---> true  [user, 1]
Adding rule:   [2] e & k & m ---> e & k  [user, 2]
Adding rule:   [3] r & t ---> r  [user, 3]
Adding rule:   [4] e & m & r ---> m & r  [user, 4]
Adding rule:   [5] e & k & r ---> (true + e + k + e & k 
                                   + r + e & r + k & r)  [user, 5]
Adding rule:   [6] c & e & m ---> false  [user, 6]
Adding rule:   [7] k & m & r ---> (k + k & m + k & r)  [user, 7]
Adding rule:   [8] e & r ---> r  [user, 8]
  Deleting rule:  [4] e & m & r ---> m & r  [user, 4]
  Deleting rule:  [5] e & k & r ---> (true + e + k + e & k + 
                                      r + e & r + k & r)  [user, 5]
Adding rule:   [9] e & k --->  (true + e + k)  [deleted, 5]
  Deleting rule:  [2] e & k & m ---> e & k  [user, 2]
Adding rule:  [10] k & m ---> (true + e + k + m + e & m)  [deleted, 2]
  Deleting rule:  [7] k & m & r ---> (k + k & m + k & r)  [user, 7]
Adding rule:  [11] e & m --->  (true + e + m)  [deleted, 7]
  Deleting rule:  [6] c & e & m ---> false  [user, 6]
  The right hand side is reducible:
     [10] k & m ---> (true + e + k + m + e & m)  [deleted, 2]
Adding rule:  [12] c & m ---> (c + c & e)  [deleted, 6]
Adding rule:  [13] c & k ---> k  [user, 9]
Adding rule:  [14] c & e & t ---> c & t  [user, 10]
Computing critical pairs with: 
    [1] a ---> true  [user, 1]
Computing critical pairs with: 
   [13] c & k ---> k  [user, 9]
Adding rule:  [15] c & e ---> (true + c + e)  [9, 13]
  The right hand side is reducible:
     [12] c & m ---> (c + c & e)  [deleted, 6]
  Deleting rule: [14] c & e & t ---> c & t  [user, 10]
Adding rule:  [16] k ---> (true + e)  [12, 13]
  Deleting rule:  [9] e & k ---> (true + e + k)  [deleted, 5]
  Deleting rule: [10] k & m ---> k  [deleted, 2]
  Deleting rule: [13] c & k ---> k  [user, 9]
Adding rule:  [17] e & t ---> t  [deleted, 14]
Computing critical pairs with: 
   [16] k ---> (true + e)  [12, 13]
Computing critical pairs with: 
   [17] e & t ---> t  [deleted, 14]
Computing critical pairs with: 
   [15] c & e ---> (true + c + e)  [9, 13]
Computing critical pairs with: 
   [12] c & m ---> (true + e)  [deleted, 6]
Computing critical pairs with: 
   [11] e & m ---> (true + e + m)  [deleted, 7]
Computing critical pairs with: 
    [8] e & r ---> r  [user, 8]
Computing critical pairs with: 
    [3] r & t ---> r  [user, 3]
Your system is canonical.
  [1] a ---> true  [user, 1]
  [3] r & t ---> r  [user, 3]
  [8] e & r ---> r  [user, 8]
 [11] e & m ---> (true + e + m)  [deleted, 7]
 [12] c & m ---> (true + e)  [deleted, 6]
 [15] c & e ---> (true + c + e)  [9, 13]
 [16] k ---> (true + e)  [12, 13]
 [17] e & t ---> t  [deleted, 14]
\end{verbatim}

\normalsize
\rm
\vspace{0.5in}
\section{Proving a First-order Formula}
\normalsize
\rm
The following example illustrates the use of \RRL for proving
first-order formula using the refutational approach. The first formula
is the hypothesis, and the second formula is the negated conclusion.
The goal is to check whether a contradiction is generated.

For proving a formula in first-order predicate calculus,
the following steps may be followed:
\begin{enumerate}
\item Give the hypotheses in a formula.
\item Give the negation of the conclusion.
\item Specify any preferred precedence relation on symbols in the formulae.
\item Invoke {\em kb}. If a contradiction is generated, then the hypotheses imply the conclusion. If \RRL stops with a complete system,
then the hypotheses do not imply the conclusion.
\end{enumerate}
Instead of giving the negation of the conclusion, it is also possible
to invoke the {\em refute} command which prompts
for a set of formulae; the last formula in the input is assumed to be
a conclusion and thus negated before {\em kb} is automatically invoked.

\scriptsize
\small
\tt
\begin{verbatim}
Type Add, Akb, Auto, Break, Clean, Delete, Dump, Grammar, Init, Kb, List,
     Log, Makerule, Narrow, Norm, Order, Option, Operator, Prove, Quit, Read,
     Refute, Stats, Suffic, Undo, Unlog, Write or Help.
\end{verbatim}
RRL-> {\em r murray1}
\begin{verbatim}
Equations read in are:
 1. all z (all y ((isequal(y, z) equ all x ((member(x, y) equ member(x, z))))))  [user, 1]
 2. not(all x (all y ((isequal(x, y) equ isequal(y, x)))))  [user, 2]
Time = 0.06666666666666288 sec
Type Add, Akb, Auto, Break, Clean, Delete, Dump, Grammar, Init, Kb, List,
     Log, Makerule, Narrow, Norm, Order, Option, Operator, Prove, Quit, Read,
     Refute, Stats, Suffic, Undo, Unlog, Write or Help.
\end{verbatim}
RRL-> {\em option fopc break-ass n}
\begin{verbatim}
Time = 0.03333333333330302 sec
Type Add, Akb, Auto, Break, Clean, Delete, Dump, Grammar, Init, Kb, List,
     Log, Makerule, Narrow, Norm, Order, Option, Operator, Prove, Quit, Read,
     Refute, Stats, Suffic, Undo, Unlog, Write or Help.
\end{verbatim}
RRL-> {\em kb}
\begin{verbatim}
----- Step 1 -----
Adding rule:   [1] (isequal(y, z) & member(x, y) + isequal(y, z) & member(x, z)) ---> false  [user, 1]
  s1(y, z) is a skolem function for x
     in the assertion: all x ((member(x, y) equ member(x, z)))
Adding rule:   [2] (member(s1(y, z), y) + member(s1(y, z), z)) ---> (true + isequal(y, z))  [user, 1]
  s2 is a skolem function for x
     in the assertion: not(all x (all y ((isequal(x, y) equ isequal(y, x)))))
  s3 is a skolem function for y
     in the assertion: not(all x (all y ((isequal(x, y) equ isequal(y, x)))))
Adding rule:   [3] (isequal(s2, s3) + isequal(s3, s2)) ---> true  [user, 2]
Computing critical pairs with: 
    [3] (isequal(s2, s3) + isequal(s3, s2)) ---> true  [user, 2]
Adding rule:   [4] (isequal(s2, s3) & member(x, s3) + isequal(s3, s2) & member(x, s2)) ---> 
                                    member(x, s2)  [1, 3]
Adding rule:   [5] (member(x, s2) + member(x, s3)) ---> false  [1, 3]
Adding rule:   [6] (isequal(s2, s3) & member(x, s2) + isequal(s3, s2) & member(x, s3)) --->  
                                    member(x, s3)  [1, 3]
Adding rule:   [7] isequal(s2, s3) & isequal(s3, s2) ---> false  [idem, 3]
Computing critical pairs with: 
    [7] isequal(s2, s3) & isequal(s3, s2) ---> false  [idem, 3]
Computing critical pairs with: 
    [5] (member(x, s2) + member(x, s3)) ---> false  [1, 3]
Adding rule:   [8] (isequal(s2, z) & member(x, z) + isequal(s2, z) & member(x, s3)) ---> false  [1, 5]
Adding rule:   [9] (isequal(s3, z) & member(x, z) + isequal(s3, z) & member(x, s2)) ---> false  [1, 5]
Adding rule:  [10] (isequal(y, s2) & member(x, y) + isequal(y, s2) & member(x, s3)) ---> false  [1, 5]
Adding rule:  [11] (isequal(y, s3) & member(x, y) + isequal(y, s3) & member(x, s2)) ---> false  [1, 5]
Adding rule:  [12] (member(s1(s2, z), z) + member(s1(s2, z), s3)) ---> (true + isequal(s2, z))  [2, 5]
Adding rule:  [13] (member(s1(s3, z), z) + member(s1(s3, z), s2)) ---> (true + isequal(s3, z))  [2, 5]
Adding rule:  [14] (member(s1(y, s2), y) + member(s1(y, s2), s3)) ---> (true + isequal(y, s2))  [2, 5]
Adding rule:  [15] (member(s1(y, s3), y) + member(s1(y, s3), s2)) ---> (true + isequal(y, s3))  [2, 5]
Adding rule:  [16] member(x, s2) & member(x, s3) ---> member(x, s2)  [idem, 5]
Computing critical pairs with: 
   [16] member(x, s2) & member(x, s3) ---> member(x, s2)  [idem, 5]
Adding rule:  [17] isequal(s3, z) & member(x, z) & member(x, s2) ---> isequal(s3, z) & member(x, s2)  [1, 16]
Adding rule:  [18] isequal(y, s2) & member(x, y) & member(x, s3) ---> isequal(y, s2) & member(x, s2)  [1, 16]
Adding rule:  [19] isequal(y, s3) & member(x, y) & member(x, s2) ---> isequal(y, s3) & member(x, s2)  [1, 16]
Adding rule:  [20] member(s1(s2, z), z) & member(s1(s2, z), s3) --->  
                           isequal(s2, z) & member(s1(s2, z), s3)  [2, 16]
Adding rule:  [21] member(s1(s3, z), z) & member(s1(s3, z), s2) --->  
                           isequal(s3, z) & member(s1(s3, z), s2)  [2, 16]
Adding rule:  [22] member(s1(y, s2), y) & member(s1(y, s2), s3) --->  
                           isequal(y, s2) & member(s1(y, s2), s3)  [2, 16]
Adding rule:  [23] member(s1(y, s3), y) & member(s1(y, s3), s2) --->  
                           isequal(y, s3) & member(s1(y, s3), s2)  [2, 16]
Adding rule:  [24] isequal(s3, z) & member(x, z) & member(x, s3) ---> isequal(s3, z) & member(x, s2)  [9, 16]
Adding rule:  [25] isequal(y, s2) & member(x, y) & member(x, s2) ---> isequal(y, s2) & member(x, s2)  [10, 16]
Adding rule:  [26] isequal(y, s3) & member(x, y) & member(x, s3) ---> isequal(y, s3) & member(x, s2)  [11, 16]
Adding rule:  [27] isequal(s2, s2) & member(s1(s2, s2), s3) ---> member(s1(s2, s2), s2)  [12, 16]
Adding rule:  [28] isequal(s2, s3) & member(s1(s2, s3), s2) ---> member(s1(s2, s3), s2)  [12, 16]
Adding rule:  [29] member(s1(s2, z), z) & member(s1(s2, z), s2) --->  
                           isequal(s2, z) & member(s1(s2, z), s2)  [12, 16]
Adding rule:  [30] isequal(s3, s2) & member(s1(s3, s2), s3) ---> member(s1(s3, s2), s3)  [13, 16]
Adding rule:  [31] isequal(s3, s3) & member(s1(s3, s3), s2) ---> member(s1(s3, s3), s2)  [13, 16]
Adding rule:  [32] member(s1(s3, z), z) & member(s1(s3, z), s3) --->  
                           isequal(s3, z) & member(s1(s3, z), s3)  [13, 16]
Adding rule:  [33] isequal(s3, s2) & member(s1(s3, s2), s2) ---> member(s1(s3, s2), s2)  [14, 16]
Adding rule:  [34] member(s1(y, s2), y) & member(s1(y, s2), s2) --->  
                           isequal(y, s2) & member(s1(y, s2), s2)  [14, 16]
Adding rule:  [35] isequal(s2, s3) & member(s1(s2, s3), s3) ---> member(s1(s2, s3), s3)  [15, 16]
Adding rule:  [36] member(s1(y, s3), y) & member(s1(y, s3), s3) --->  
                           isequal(y, s3) & member(s1(y, s3), s3)  [15, 16]
Computing critical pairs with: 
   [35] isequal(s2, s3) & member(s1(s2, s3), s3) ---> member(s1(s2, s3), s3)  [15, 16]
Adding rule:  [37] isequal(y, s3) & isequal(s2, s3) & member(s1(s2, s3), y) --->  
                           isequal(y, s3) & member(s1(s2, s3), s3)  [1, 35]
Adding rule:  [38] isequal(s2, s3) ---> true  [2, 35]
  Deleting rule:  [3] (isequal(s2, s3) + isequal(s3, s2)) ---> true  [user, 2]
  Deleting rule:  [4] (isequal(s2, s3) & member(x, s3) + isequal(s3, s2) & member(x, s2)) --->  
                           member(x, s2)  [1, 3]
  Deleting rule:  [6] (isequal(s2, s3) & member(x, s2) + isequal(s3, s2) & member(x, s3)) --->  
                           member(x, s3)  [1, 3]
  Deleting rule:  [7] isequal(s2, s3) & isequal(s3, s2) ---> false  [idem, 3]
  Deleting rule: [28] isequal(s2, s3) & member(s1(s2, s3), s2) ---> member(s1(s2, s3), s2)  [12, 16]
  Deleting rule: [35] isequal(s2, s3) & member(s1(s2, s3), s3) ---> member(s1(s2, s3), s3)  [15, 16]
  Deleting rule: [37] isequal(y, s3) & isequal(s2, s3) & member(s1(s2, s3), y) --->  
                           isequal(y, s3) & member(s1(s2, s3), s3)  [1, 35]
Adding rule:  [39] isequal(s3, s2) ---> false  [deleted, 7]
  Deleting rule: [30] isequal(s3, s2) & member(s1(s3, s2), s3) ---> member(s1(s3, s2), s3)  [13, 16]
  Deleting rule: [33] isequal(s3, s2) & member(s1(s3, s2), s2) ---> member(s1(s3, s2), s2)  [14, 16]
Adding rule:  [40] member(s1(s3, s2), s2) ---> false  [deleted, 33]
Adding rule:  [41] member(s1(s3, s2), s3) ---> false  [deleted, 30]
Computing critical pairs with: 
   [39] isequal(s3, s2) ---> false  [deleted, 7]
Computing critical pairs with: 
   [38] isequal(s2, s3) ---> true  [2, 35]
Computing critical pairs with: 
   [41] member(s1(s3, s2), s3) ---> false  [deleted, 30]
Adding rule:  [42] isequal(s3, z) & member(s1(s3, s2), z) ---> false  [1, 41]
Adding rule:  [43] isequal(y, s3) & member(s1(s3, s2), y) ---> false  [1, 41]
Derived:  [44] true ---> false  [2, 41]
Your system is not consistent.
Processor time used                = 16.47 sec
Number of rules generated          = 44
Number of critical pairs           = 98
Time spent in normalization        = 9.53 sec     (57.89 percent of time)
Time spent while adding rules      = 1.43 sec     (8.70 percent of time)
  (keeping rule set reduced)
Time spent in unification          = 0.37 sec     (2.23 percent of time)
Time spent in ordering             = 0.88 sec     (5.36 percent of time)
Time spent in boolean translation  = 6.03 sec     (36.64 percent of time)
Total time (including 'undo' action) = 16.47 sec
Time = 16.48333333333333 sec
\end{verbatim}
\rm
\normalsize
\vspace{0.5in}
\section{Illustration of Paramodulation}
\rm
The simple example below illustrates the need for handling formula
involving equality relation when they are part of a larger formula.
RRL has a special option for turning the paramodulation-like facility
using {\em option fopc paramodulation yes}, which is generally turned
off because it generates too many rules. This example also illustrates
that in the default mode, the theorem proving method in RRL is {\bf
incomplete} for first-order predicate calculus with equality. However,
with the paramodulation option set, the theorem proving method is {\bf
complete} for first-order predicate calculus with equality.

For proving a formula expressed using the equality predicate,
it is recommended that the user first try RRL without turning
on the paramodulation option as many formulae can be proved 
in this way. If it appears that a formula cannot be proved,
then the paramodulation option can be turned on and the proof
of the formula retried.

\small
\tt
\begin{verbatim}
Type Add, Akb, Auto, Break, Clean, Delete, Dump, Grammar, Init, Kb, List,
     Log, Makerule, Narrow, Norm, Order, Option, Operator, Prove, Quit, Read,
     Refute, Stats, Suffic, Undo, Unlog, Write or Help.
\end{verbatim}
RRL-> {\em a}
\begin{verbatim}
Type your equations, enter a ']' when done.
\end{verbatim}
{\em eq(a, b) or eq(c,d)\\
eq(a, c) or eq(b, d)\\
not (eq(a, d) or eq(b, c))\\
] }
\begin{verbatim}
Equations read in are:
 1. (eq(a, b) or eq(c, d))  [user, 1]
 2. (eq(a, c) or eq(b, d))  [user, 2]
 3. not((eq(a, d) or eq(b, c)))  [user, 3]
New constant set is: { d, c, b, a }
Time = 0.09999999999999964 sec
Type Add, Akb, Auto, Break, Clean, Delete, Dump, Grammar, Init, Kb, List,
     Log, Makerule, Narrow, Norm, Order, Option, Operator, Prove, Quit, Read,
     Refute, Stats, Suffic, Undo, Unlog, Write or Help.
\end{verbatim}
RRL-> {\em kb}
\begin{verbatim}
----- Step 1 -----
Adding rule:   [1] eq(a, b) & eq(c, d) ---> 
               (true + 
                eq(a, b) + 
                eq(c, d))  [user, 1]
Adding rule:   [2] eq(a, c) & eq(b, d) ---> 
               (true + 
                eq(a, c) + 
                eq(b, d))  [user, 2]
Adding rule:   [3] eq(a, d) ---> false  [user, 3]
Adding rule:   [4] eq(b, c) ---> false  [user, 3]
Computing critical pairs with: 
    [4] eq(b, c) ---> false  [user, 3]
Computing critical pairs with: 
    [3] eq(a, d) ---> false  [user, 3]
Computing critical pairs with: 
    [2] eq(a, c) & eq(b, d) ---> 
               (true + 
                eq(a, c) + 
                eq(b, d))  [user, 2]
Computing critical pairs with: 
    [1] eq(a, b) & eq(c, d) ---> 
               (true + 
                eq(a, b) + 
                eq(c, d))  [user, 1]
Your system is canonical.
  [1] eq(a, b) & eq(c, d) ---> 
               (true + 
                eq(a, b) + 
                eq(c, d))  [user, 1]
  [2] eq(a, c) & eq(b, d) ---> 
               (true + 
                eq(a, c) + 
                eq(b, d))  [user, 2]
  [3] eq(a, d) ---> false  [user, 3]
  [4] eq(b, c) ---> false  [user, 3]
Type Add, Akb, Auto, Break, Clean, Delete, Dump, Grammar, Init, Kb, List,
     Log, Makerule, Narrow, Norm, Order, Option, Operator, Prove, Quit, Read,
     Refute, Stats, Suffic, Undo, Unlog, Write or Help.
\end{verbatim}
RRL-> {\em undo}
\begin{verbatim}
----- The original system is restored -----
Equations:
 1. (eq(a, b) or eq(c, d))  [user, 1]
 2. (eq(a, c) or eq(b, d))  [user, 2]
 3. not((eq(a, d) or eq(b, c)))  [user, 3]
No rules in current system.
Time = 0.01666666666666661 sec
Type Add, Akb, Auto, Break, Clean, Delete, Dump, Grammar, Init, Kb, List,
     Log, Makerule, Narrow, Norm, Order, Option, Operator, Prove, Quit, Read,
     Refute, Stats, Suffic, Undo, Unlog, Write or Help.
\end{verbatim}
RRL-> {\em opt fopc para}
\begin{verbatim}
Use paramodulation (y) or not (fast & incomplete) (n) ?
\end{verbatim}
(current value: n) (y,n) {\em y}
\begin{verbatim}
Time = 0.03333333333333321 sec
Type Add, Akb, Auto, Break, Clean, Delete, Dump, Grammar, Init, Kb, List,
     Log, Makerule, Narrow, Norm, Order, Option, Operator, Prove, Quit, Read,
     Refute, Stats, Suffic, Undo, Unlog, Write or Help.
\end{verbatim}
RRL-> {\em kb}
\begin{verbatim}
----- Step 1 -----
Adding rule:   [1] eq(a, b) & eq(c, d) ---> 
               (true + 
                eq(a, b) + 
                eq(c, d))  [user, 1]
Adding rule:   [2] eq(a, c) & eq(b, d) ---> 
               (true + 
                eq(a, c) + 
                eq(b, d))  [user, 2]
Adding rule:   [3] eq(a, d) ---> false  [user, 3]
Adding rule:   [4] eq(b, c) ---> false  [user, 3]
Computing critical pairs with: 
    [4] eq(b, c) ---> false  [user, 3]
Adding rule:   [5] eq(a, c) ---> false  [1, 4]
  Deleting rule:  [2] eq(a, c) & eq(b, d) ---> 
               (true + 
                eq(a, c) + 
                eq(b, d))  [user, 2]
Adding rule:   [6] eq(b, d) ---> false  [1, 4]
Derived:   [7] true ---> false  [deleted, 2]
Your system is not consistent.
\end{verbatim}
\rm
\normalsize

