\chapter {Illustration of inductionless induction} 
\normalsize 
\rm 
The inductionless induction or proof-by-consistency method for proving
equations by induction is illustrated. Addition ($+$) and
multiplication ($*$) are defined on natural numbers, and their
additional properties are specified which can also be proved by
induction. Then, another function $sum$, to stand for
summation from $0$ to $x$, is also recursively defined.  The goal is
to show that $sum(x) * s(s(0))$ is indeed $(x * x) + x$. 

The following steps may be followed to prove an equation by induction
using the proof-by-consistency method:
\begin{enumerate}
\item Give an axiomatization as input.
\item Generate a complete rewriting system from the input.
\item Attempt to prove the equation under consideration. If
it is not in the equational theory, RRL will prompt asking whether
the equation should be tried by induction. Respond with {\em y}.
\item Specify the constructors of the domain on which induction is being
performed. RRL will check whether the remaining functions are completely
defined, the condition that must be satisfied for proof-by-consistency
method to be applicable.
\item If the functions are completely defined, then RRL will attempt
to check for the consistency of the original theory augmented
with the equation being proved. This is done by attempting to generate a
complete system and ensuring that no equation relating unequal constructor
values is generated.
\item If an inconsistency is generated, then the complete
system of the original theory is restored.
\item If no inconsistency is generated, then
the user is given the choice of adding the proved equation to the
complete system of the original theory or restoring the complete
system of the original theory.
\end{enumerate}
Often, while proving an equation by induction, a potentially
infinite set of rules with similar structure is generated; these rules
usually suggest a new lemma generalizing these rules, which should
be proved before the original equation can be proved.

\small
\tt
\begin{verbatim}
Type Add, Akb, Auto, Break, Clean, Delete, Dump, Grammar, Init, Kb, List,
     Log, Makerule, Narrow, Norm, Order, Option, Operator, Prove, Quit, Read,
     Refute, Stats, Suffic, Undo, Unlog, Write or Help.
\end{verbatim}
RRL-> {\em induction1}
\begin{verbatim}
Equations read in are:
 1. (x + 0) == x  [user, 1]
 2. (x + s(y)) == s((x + y))  [user, 2]
 3. (x * 0) == 0  [user, 3]
 4. (x * s(y)) == ((x * y) + x)  [user, 4]
 5. sum(0) == 0  [user, 5]
 6. sum(s(x)) == (sum(x) + s(x))  [user, 6]
New constant set is: { 0 }
Time = 0.05000000000000027 sec
Type Add, Akb, Auto, Break, Clean, Delete, Dump, Grammar, Init, Kb, List,
     Log, Makerule, Narrow, Norm, Order, Option, Operator, Prove, Quit, Read,
     Refute, Stats, Suffic, Undo, Unlog, Write or Help.
\end{verbatim}
RRL-> {\em oper ac +}
\begin{verbatim}
'+' is associative & commutative now.
Time = 0.04999999999999982 sec
Type Add, Akb, Auto, Break, Clean, Delete, Dump, Grammar, Init, Kb, List,
     Log, Makerule, Narrow, Norm, Order, Option, Operator, Prove, Quit, Read,
     Refute, Stats, Suffic, Undo, Unlog, Write or Help.
\end{verbatim}
RRL-> {\em oper comm *}
\begin{verbatim}
'*' is commutative now.
Time = 0.01666666666666705 sec
Type Add, Akb, Auto, Break, Clean, Delete, Dump, Grammar, Init, Kb, List,
     Log, Makerule, Narrow, Norm, Order, Option, Operator, Prove, Quit, Read,
     Refute, Stats, Suffic, Undo, Unlog, Write or Help.
\end{verbatim}
RRL-> {\em kb}
\begin{verbatim}
----- Step 1 -----
Adding rule:   [1] (x + 0) ---> x  [user, 1]
Trying to orient equation: 
  (x + s(y)) == s((x + y))  [user, 2]
  To prove:  (x + s(y))  >  s((x + y))
        or:  (x + s(y))  <  s((x + y))
  Here are some precedence suggestions:
       1.  + > s                     2.  s > +
Either type a list of numbers or
Type Abort, Display, Equiv, Twoway, Operator, Status, LR, RL, Postpone, MakeEq,
     Quit, Undo or Help.
\end{verbatim}
RRL>KB> {\em lr}
\begin{verbatim}
Orientation of equations being done manually.
!!!!!! Warning: Rewriting may not terminate !!!!
!!!!!! Warning: Resulting system will be locally-confluent;
         But it may not be canonical  !!!!
\end{verbatim}
Is it ok to continue ? (y,n) {\em y}
\begin{verbatim}
----- Step 2 -----
Adding rule:   [2] (x + s(y)) ---> s((x + y))  [user, 2]
Adding rule:   [3] (x * 0) ---> 0  [user, 3]
Trying to orient equation: 
  (x * s(y)) == (x + (x * y))  [user, 4]
  To prove:  (x * s(y))  >  (x + (x * y))
        or:  (x * s(y))  <  (x + (x * y))
  Here are some precedence suggestions:
       1.  * > +                     2.  + > *
       3.  s > +                     4.  + > s
       5.  s > *                     6.  * > s
Either type a list of numbers or
Type Abort, Display, Equiv, Twoway, Operator, Status, LR, RL, Postpone, MakeEq,
     Quit, Undo or Help.
\end{verbatim}
RRL>KB> {\em 1}
\begin{verbatim}
----- Step 3 -----
Adding rule:   [4] (x * s(y)) ---> (x + (x * y))  [user, 4]
Adding rule:   [5] sum(0) ---> 0  [user, 5]
Trying to orient equation: 
  sum(s(x)) == s((x + sum(x)))  [user, 6]
  To prove:  sum(s(x))  >  s((x + sum(x)))
        or:  sum(s(x))  <  s((x + sum(x)))
  Here are some precedence suggestions:
       1.  sum > s                   2.  s > sum
       3.  sum > +                   4.  + > sum
       5.  s > +                     6.  + > s
Either type a list of numbers or
Type Abort, Display, Equiv, Twoway, Operator, Status, LR, RL, Postpone, MakeEq,
     Quit, Undo or Help.
\end{verbatim}
RRL>KB> 1 3
\begin{verbatim}
----- Step 4 -----
Adding rule:   [6] sum(s(x)) ---> s((x + sum(x)))  [user, 6]
Your system is locally confluent.
  [1] (x + 0) ---> x  [user, 1]
  [2] (x + s(y)) ---> s((x + y))  [user, 2]
  [3] (x * 0) ---> 0  [user, 3]
  [4] (x * s(y)) ---> (x + (x * y))  [user, 4]
  [5] sum(0) ---> 0  [user, 5]
  [6] sum(s(x)) ---> s((x + sum(x)))  [user, 6]
Time = 0.8666666666666663 sec.
 ... ...
Type Add, Akb, Auto, Break, Clean, Delete, Dump, Grammar, Init, Kb, List,
     Log, Makerule, Narrow, Norm, Order, Option, Operator, Prove, Quit, Read,
     Refute, Stats, Suffic, Undo, Unlog, Write or Help.
\end{verbatim}
RRL-> {\em prove}
\begin{verbatim}
Type equation to prove in the format:  L == R (if C) 
Enter a ']' to exit when no equation is given.
\end{verbatim}
{\em sum(x) * s(s(0)) == (x * x) + x}
\begin{verbatim}
No, it is not equational theorem.
Normal form of the left hand side is:
(sum(x) + sum(x))

Normal form of the right hand side is:
((x * x) + x)
\end{verbatim}
Do you want to see it is an inductive theorem ? {\em y}
\begin{verbatim}
----- Step 5 -----
      Current Constructor Set  =  { }
\end{verbatim}
To prove the equation with the constructors ? (y,n) {\em y}
\begin{verbatim}
Note: Constructor set is empty.
\end{verbatim}
Type operators you wish to be constructor: {\em 0 s}
\begin{verbatim}
    Constructor Set =  { s, 0 }
Specification of 'sum' is complete relative to { s, 0 }
Specification of '*' is complete relative to { s, 0 }
Specification of '+' is complete relative to { s, 0 }
Processor time used      =  0.09999999999999964 sec
Proving equation
  (sum(x) * s(s(0))) == ((x * x) + x)  [user, 7]
Trying to orient equation: 
  (sum(x) + sum(x)) == (x + (x * x))  [user, 7]
  To prove:  (sum(x) + sum(x))  >  (x + (x * x))
        or:  (sum(x) + sum(x))  <  (x + (x * x))
  Here are some precedence suggestions:
       1.  sum > *                   2.  * > sum
Either type a list of numbers or
Type Abort, Display, Equiv, Twoway, Operator, Status, LR, RL, Postpone, MakeEq,
     Quit, Undo or Help.
\end{verbatim}
RRL>KB> {\em 1}
\begin{verbatim}
----- Step 6 -----
Adding rule:   [7] (sum(x) + sum(x)) ---> (x + (x * x))  [user, 7]

Your system is locally confluent.
  [1] (x + 0) ---> x  [user, 1]
  [2] (x + s(y)) ---> s((x + y))  [user, 2]
  [3] (x * 0) ---> 0  [user, 3]
  [4] (x * s(y)) ---> (x + (x * y))  [user, 4]
  [5] sum(0) ---> 0  [user, 5]
  [6] sum(s(x)) ---> s((x + sum(x)))  [user, 6]
  [7] (sum(x) + sum(x)) ---> (x + (x * x))  [user, 7]

Following equation
  (sum(x) * s(s(0))) == ((x * x) + x)  [user, 7]
  is an inductive theorem in the current system.
\end{verbatim}
\rm
\normalsize
