\section{The Knuth-Bendix Completion Procedure: \em kb}

The command {\em kb} invokes the extended completion procedure on the
existing system which in general is a combination of rules and
equations.  Depending upon the properties of function symbols
appearing in the rules and equations, and structure of rules and
equations, different completion procedures are invoked.  If some
logical formula is included in the current system, then the completion
procedure for first-order predicate calculus is invoked.  If there
exist some conditional equations or rules in the current system, then
the conditional completion procedure is applied.
  
\subsection{Behaviors of the Completion Procedure}

Invoking {\em kb} can result
in one of the following possibilities:
\begin{enumerate}
\item \RRL stops with a set of rules. In this case,
if the resulting set of rules includes
{\tt true -> false}, then the input axioms are
inconsistent. If the rule {\tt true -> false} is not in
the result and none of the operators is specified
to be associative-commutative, the resulting
system is a complete set of rules,
which is a decision procedure for the input
axioms. However, if some operators are declared
to be associative-commutative, then an AC-ordering
is applied when some conditions are satisfied. 
If these conditions are not satisfied, then the result of {\em kb} is a
locally-confluent set of rules, as the
result is not guaranteed to be a terminating set of rules.
The termination algorithm used
in \RRL does not work for many equations using
associative-commutative operators,
although our experience suggests that for many associative-commutative 
examples, termination of the rules can be easily ensured.
\item During the running of the completion
procedure, an equation may be encountered that cannot
be oriented into a rule because of reasons
discussed in the next subsections. In that case, the user has many
options: 
\begin{itemize}
\item quit and retry using a different
termination ordering (see below), 
\item  orient
equations by hand, in which case, if \RRL stops
with a set of rules, the result is a locally-confluent \cite{Huet80}
system as \RRL does not guarantee that
the rules terminate, or
\item  undo previous choices made about making rules from equations
until an \RRL state is reached
from where a desired choice can be
made.
\end{itemize}
\item \RRL keeps generating new rules, until
interrupted by the user by {\em $\uparrow$-C} on UNIX systems and by
the {\em abort} (or {\em $\uparrow$-abort}) key on the Lisp machine.
On UNIX, the user is then talking to Lisp; typing {\em $\uparrow$-D}
or {\em (rrl)} will bring the user back to \ERRL.  It is also possible
to specify the maximum run time for which \RRL should be run on a
problem; this is done using the command {\em option/brake/runtime}.
When the specified maximum run time is elapsed,
\RRL will stop. Then the user may continue by increasing the run time
limit or quit the
completion procedure.
\end{enumerate}  
Appendix C is a transcript of invoking {\em kb} on an equational
axiomatization of free groups. Appendix E is a transcript of invoking
{\em kb} on an axiomatization of free abelian groups.

\subsection{Introducing New Operators}

An interesting feature of the completion procedure is that it can
generate an equation which cannot be made into a rewrite rule. This
happens if both sides of an equation have a variable(s) whose value
can be arbitrarily chosen without affecting the validity of the
equation. A simple example of such an equation is:
\begin{center}
\tt   x / x == y / y.
\end{center}
\RRL then prompts the user with the following
message:
\begin{verbatim}
This eqn cannot be oriented into a rule:
  (x / x) == (y / y)  [3, 4]
\end{verbatim}
\tt 
Type left\_right, makeeq, operator, postpone, right\_left, undo, quit, help -->\\
\rm
giving the user an option of either
\begin{itemize}
\item
to postpone the
equation for consideration later with the hope
that it would get simplified and the result
can be made into a rule, or

\item to orient the equation by hand (either left to right or right to left);
then, the rewrite rule will not be used for rewriting, or,

\item to make the rewrite rule $eq(t_1 , t_2) \rightarrow true$
from the equation $t_1 = t_2$, or

\item introduce a new
operator capturing what variables the value of
each side really depends upon.

\end{itemize}  

In case a new operator symbol is to be introduced, this is illustrated below:\\\
\tt 
Type left\_right, makeeq, operator, postpone, right\_left, undo, quit, help --> {\em operator}\\
----- Step 4 -----\\
Give me a new operator name: {\em e}
\begin{verbatim}
Trying to orient equation: 
  (x / x) == e  [3, 4]
\end{verbatim}
\rm

The feature of introducing new operators
turns out to be extremely useful. It has been
successfully employed to show the equivalence of different
axiomatizations of free groups, for example; see Appendix D.

\section{Orienting Equations as Rules: {\em kb, akb, makerule}}
An integral part of the completion procedure is a way to orient
equations into rewrite rules.  This is done either interactively with the
user's help or in a totally automatic mode. When the command {\em kb}
is invoked, equations are oriented in an interactive mode. The command
{\em akb} supports the automatic mode for orienting equations into
rules.
  
In the interactive mode, rules can be made from equations manually
or using an algorithm for {\em recursive path ordering} ({\em
rpo})\cite{Dershowitz82} and {\em lexicographic} recursive path
ordering ({\em lrpo}) \cite{KaminLevy}\cite{Dershowitz87}.  We have
found {\em rpo} and {\em lrpo} efficient and fairly easy to implement.  Both of
them extend a precedence relation on function symbols to terms.  As
discussed above, for interactive as well as automatic modes, an
initial precedence relation on functions and their statuses can be
specified using the operator command before invoking {\em kb} or {\em
akb}.
  
The ordering {\em rpo} possesses the incrementality property in the sense that
the precedence relation on function symbols can be incrementally
specified as the completion procedure proceeds.  In {\em rpo}, every
function has a multiset status meaning that every argument to the
function is given equal weight.  In the case of {\em lrpo}, when an operator
is incrementally set to have a left-to-right or right-to-left
status, rules already oriented have to be checked again to make sure
that the orientation is consistent with the status. If a proposed
extension to the precedence relation is inconsistent, \RRL notifies the
user of that error.

\subsection{Interactive Mode}

Whenever an equation to be made into a rule
cannot be oriented by the algorithm using the
existing precedence relation, \RRL presents the
user with different possible ways to handle that case:
\begin{enumerate}
\item enhance the precedence, that is,
add some relation on operators to the precedence; 
\item change the status of some operators; 
\item introduce a new operator;
\item use the command {\em undo} to cancel a previous choice;
\item interrupt and start with a new ordering using 
whatever is in the rule set and equation set;
\item orient equations into rules manually - {\em lr}, left-to-right, 
or {\em rl}, right-to-left.
\end{enumerate}

A typical \RRL message in this case looks like:
\begin{verbatim}
 ... ...
Trying to orient equation: 
  (i(x) * x) == e  [user, 2]

  To prove:  (i(x) * x)  >  e
  Here are some precedence suggestions:
       1.  * > e
       2.  i > e
Either type a list of numbers or
Type Abort, Display, Drop, Equiv, LR, MakeEq, Operator, Postpone, Quit, RL,
     Status, Superpose, Twoway, Undo or Help.
RRL>KB> 
\end{verbatim}
  
The user is asked to make one of the six choices listed above, or to
make two operators to have equivalent precedence and status. All
possible extensions of the precedence relation are presented.  There
is no guarantee that each of the extensions is likely to work; in this
sense, the algorithm for suggesting precedence does not try to be
smart.  If the proposed extension to the precedence relation is
inconsistent with the existing precedence or cannot orient the rule,
the user is notified.  It is possible to display the current
precedence relation on functions and their status using the {\em
display} command at this stage.

It is possible that the equation may not be orientable no matter how
the precedence relation is extended.  In such a situation, it is
better to 
\begin{itemize}
\item postpone the equation with the hope that it will get
simplified later with other rules and the simplified form can be
oriented, 
\item introduce a new operator, or 
\item undo to a previous
choice to make an alternative choice.
\end{itemize}

For the user who knows {\em rpo} or {\em lrpo}, the interactive mode is more
effective.  In most examples discussed in the Appendices, equations
are interactively orienting into rules by specifying precedence
relation on function symbols.

The first time the user chooses to orient an equation manually,
\RRL prompts with the following
warning.
\begin{verbatim}
 ... ...
Type Abort, Display, Drop, Equiv, LR, MakeEq, Operator, Postpone, Quit, RL,
     Status, Superpose, Twoway, Undo or Help.
\end{verbatim}
\tt RRL>KB> {\em lr}
\begin{verbatim}
Orientation of equations being done manually.

!!!!!! Warning: Rewriting may not terminate !!!!
!!!!!! Warning: Resulting system will be locally-confluent;
         But it may not be canonical  !!!!

Is it ok to continue (y,n)?
\end{verbatim}
\rm  
Equations should be manually oriented only if rules can be proved to
be terminating. \RRL provides a limited check for a possible loop in
rewriting; if a term is rewritten quite a lot, then 
\RRL stops with a warning
of a possible loop in rewriting. The maximal number of rewritings
in a normalization is 1000 by default; this number can be reset
by the user using the command {\em option/brake/normalize}.

The command {\em makerule} can be used to
make rules from equations without invoking the completion procedure.
The result is a terminating rewriting system.
This command is useful to support other methods like
narrowing, explicit induction or computing by rewriting, 
where a canonical rewriting system may not exist.
Appendix K includes an example: A
recursive definition of Ackermann's function is given
and is transformed into a terminating rewriting system using
{\em makerule}; its execution on two inputs is illustrated, too.

\subsection{Automatic Mode}

In the automatic mode, \RRL itself makes choices when an equation
cannot be oriented into a rule using the existing precedence relation.
Starting with the initial precedence, \RRL tries to find a precedence
under which each rule is terminating. The method used is called 
{\em automatic ordering with bounded depth-first search} ({\em autoorder});
see also \cite{DetlefsForgaard}.
  
There are several options for
autoorder as the following message indicates:\\ \\
\tt
RRL-> {\em option autoorder}
\begin{verbatim}
Type eq-pre, lr-rl, new-oper, postpone, keep-rule, post-limit, rule-limit, 
     quit, help --> 
\end{verbatim}
\rm
\begin{enumerate}
\item If two function symbols are not comparable
in the existing precedence relation, it is possible
to relate the precedence of the function symbols in two ways:
\begin{enumerate}
\item make the function symbols have the same precedence at first.
\item make one function symbol to have greater precedence than
the other function symbol at first.
\end{enumerate}
The command to set this option is: {\em option/autoorder/eq-pre}.
The default option is to make the function symbols have equivalent
precedence first.
\item When an operator (whose arity is bigger than 1) has no status (i.e.,
it has multi-set status meaning that all arguments are treated equally),
there are two options in giving it a status: 
\begin{enumerate} 
\item left-to-right
status first;
\item right-to-left status at first.
\end{enumerate}
The command to set this option is: {\em option/autoorder/lr-rl}.
The default option is to try right-to-left status first.
\item It is always possible to introduce a new operator for
an unoriented equation. However this can be done
\begin{enumerate}
\item before extending the existing
precedence relation; 
\item after trying extending 
precedence but before trying assigning status and
\item finally, after trying extending precedence and assigning status.
\end{enumerate}
The command to set this option is: {\em option/autoorder/new-oper}.
The default is (i), to introduce a new operator as soon as possible.
\item 
It is always possible to postpone an unoriented equation:
\begin{enumerate}
\item either without first extending the precedence relation; 
\item or, after having tried to extend the precedence but before trying to
assign status; 
\item or, after having tried to extend the precedence
and status. 
\end{enumerate}  
If choices 3 and 4 can both be made, then 3 is preferred
over 4.

The command to set this option is: {\em option/autoorder/postpone}.
The default is to postpone an equation after extending precedence but before
assigning status to an operator.
\item In general, some new rules are
generated when the precedence is
changed. If we want to cancel the previous choices, then we have two choices:
keep the new generated rules in the system as 
equations (since they are valid consequences) or not.

The command to set this option is: {\em option/autoorder/keep-rule}.
The default is not to keep the new generated rules in the system because
it has been found in practice that in many cases keeping
these rules does not help the completion procedure in generating
a complete system.
\end{enumerate}
  
Besides the above choices, there are two other parameters in {\em autoorder}: 
one is the maximum number of equations that can be postponed; the other is
the maximum number of rules that can added to the system without
changing the precedence relation. The first bound prevents \RRL from
infinitely postponing equations; the second bound prevents \RRL from
infinitely generating rules without changing the precedence.
The commands to set these bounds are: {\em
option/autoorder/post-limit} and {\em option/autoorder/rule-limit}.
The default value for post-limit is 5 and for rule-limit is 50.

The default values for various options for the automatic mode of the
termination algorithm have been arrived after experimenting with many
examples in algebra. These values may not be optimal in other
application areas.
  
Orienting equations without any assistance from the user using the
{\em akb} command is illustrated in Appendix D in which a complete
system is generated from single equation axiomatization of free groups
due to Higman. In this example, \RRL discovers new operators for
identity and inverse.

\subsection{Conditional Rewrite Rules \label{condi}}  

In cases of proof by refutation using conditional rewrite rules 
or proof by induction using the cover set method, the user must
input clauses or (conditional) equations to \ERRL. For example,
the following two formulas are acceptable by these 
two methods:
\begin{verbatim}
            p(x) if q(x) and r(x)
            p(x) or not q(x) or not r(x)
\end{verbatim}
while the following input is illegal to these two methods:
\begin{verbatim}
            all x (p(x) or not q(x) or not r(x))
            p(x) or not (q(x) and r(x))
\end{verbatim}
even though all the above four formulas are logically equivalent; however, 
they are acceptable by the method based on the boolean ring approach.

Clauses are always written in the form of conditional equations in \ERRL.
Given a conditional equation, \RRL chooses a maximal (under a given
ordering) literal
appearing in it as the body of a conditional rewrite rule and 
puts the remaining literals into the condition of the rule.
For example, for the equation {\tt p(x) if q(x) and r(x)},
if {\tt p > q > r}, then the rule is 
\begin{verbatim}
            p(x) ---> true if q(x) and r(x).
\end{verbatim}
If the order is {\tt r > q > q}, then the rule is 
\begin{verbatim}
            r(x) ---> false if q(x) and not p(x).
\end{verbatim}
If more than one literal are maximal, \RRL chooses the first
maximal literal (counted from left to right) as the body and
only one rule is made from the given equation; this is in contrast
to a statement made in \cite{ZK88} that ``more than one rule is
made from a clause if there are more than maximal literals in
the clause.'' Such rules are not used for rewriting.
Note also that the condition of a rule may contain variables 
which do not appear in the body of the rule.

\section{Parameters to the Completion Procedure}

The completion procedure has three strategies as
its parameters.
\begin{itemize}
     \item Normalization strategy;
     \item Critical pair strategy;
     \item Strategy for picking next equation for processing.
\end{itemize}
These strategies are different
for different completion procedures and are described
in the following subsections.

\subsection{Normalization Strategy: {\em option/norm, norm}}

\subsubsection{Unconditional Rewriting}

Four methods of normalizing a term with respect to a set of
unconditional rules are supported.  For discussing different
normalization methods, it is necessary to refer to subterms in
a term.  Using a tree representation of a term, its subterms can be
uniquely identified by {\em positions} which are sequence of positive
integers (see \cite{KMN84} for a definition of positions
for flattened terms when associative-commutative operators appear in
terms). A position of a term can also be used to uniquely identify
every node in its tree representation. 
The position of the empty
sequence corresponds to the root of the tree representing the term.
A position $p_1$ corresponds to
a node that is
the parent of another node identified by
another position $p_2$ if and only if $p_2 = p_1 \cdot i$, where
$i$ is a positive integer less than or equal to the arity of 
the outermost function
symbol of the subterm at $p_1$.
\begin{description} 
\item{\bf Leftmost-outermost:} This strategy
attempts to apply a rule as near the root of a term as possible. If
the rewriting at the root is not possible, then the left-most child of
the root is completely explored before we get to the other children of
the root.  

\item{\bf Leftmost-innermost:} In this strategy, a term is
explored in depth-first order while locating subterms at which
rewriting may be possible. A parent position
is processed only after its children have been explored. That is, if
we are trying to match a rule at position $p$, we must have already
normalized the children of $p$. It is as if we traverse up from the
leaves applying a rule at the first rewritable position. 

\item{\bf Efficient Innermost:} This is essentially an improvement on
leftmost-innermost strategy.  In the leftmost-innermost strategy, we
do not utilize the information that whenever we rewrite at a position
$p$, the substitution for the variables are already normalized and we do
not have to start again from the leaves to ensure the term is
normalized. This is done by keeping the bindings of the variables
separate and not replacing them in the term when we do a rewrite.

\item{\bf Efficient Outermost:} This is
a mixture of the leftmost-outermost strategy and efficient innermost
strategy. It is motivated by the fact that the main goal of
reduction is to normalize
a term and not just to rewrite it in a single step.
The leftmost-outermost strategy is used first to rewrite the
term from the root to the leaves. Then the efficient innermost
strategy is used to go up if some children of the root are reduced.
\end{description}

The distinction between the two outermost
strategies can be illustrated by 
a simple example: Suppose
$R$ contains
\[\begin{array}{rlcl}
  1. & f(g(x)) & \rightarrow & x, \\
  2. & h(x) & \rightarrow & g(x),
\end{array}\]
and the term to be normalized is $t = f(h(a))$.  $R$ is matched first
against the whole term $t$ (i.e. at its root position), and then at
the position 1 of $t$ in both strategies.  The subterm $h(a)$
can be reduced to $g(a)$ by Rule 2, so the subterm of $t$ at position
1 is replaced by $g(a)$. The rewriting is continued from the subterm
at position 1 onwards instead of going back to the root of the new
term.  $R$ is matched against subterms at positions 1 and 1.1 and it
is found that $g(a)$ is in normal form.  The
rewriting is tried once again from the root downwards.  Now $R$ can
match against $f(g(a))$ by Rule 1 at the root position.  The result of
this rewriting is $g(a)$ with the substitution $\sigma = \{ x/g(a)
\}$.  In the efficient outermost strategy, because $\sigma(x) = g(a)$
is already known to be in normal form,
the rewriting process stops.
In the leftmost-outermost strategy, it is checked that $g(a)$ is in normal
form. The distinction between efficient innermost and leftmost innermost
is similar.

Each of the four normalization methods for unconditional rewriting is
complete in the sense that at the end, the result is in normal form
if the set of rules is terminating.  They affect only the time spent
in normalization.  However, in the presence of commutative, or
associative-commutative operators, only the first rewriting strategy
(leftmost-outermost) is supported.

\subsubsection{Conditional Rewriting}

Only one strategy, named ``leftmost-innermost'',
is supported for this kind of conditional rewriting.
However, \RRL provides an option to set the {\em depth} of
conditional rewriting.
{\em depth} is associated with conditions of rewrite rules,
because conditional rewriting is recursively defined
as ``a conditional rewrite rule can reduce a term if the condition
of the rule can be {\em reduced} to be true''.
We illustrate the notion of 
the {\em depth} by a simple example: Given two rules
\[\begin{array}{rlll}
1. & x \leq x & \rightarrow & true, \\
2. & x \leq s(y) & \rightarrow & true~{\bf if}~ x \leq y,
\end{array}\]
the depth of rewriting $0 \leq 0$ to $true$ is 1 because 
the rule 1 is unconditional. 
The depth of rewriting 
$0 \leq s(0)$ to $true$ 
is 2 because the rule 2 applies first, and the depth of
rewriting its condition to true is 1.
Similarly, the depth of rewriting $0 \leq s^i (0)$ to $true$ is $i~+~1$
because the rule 2 applies first, and the depth of
rewriting its condition to true is $i$.

In some examples, the depth of a rewriting can be infinity. 
For instance, using the rule 
\[ a \rightarrow b ~{\bf if}~ a \neq c\]
to normalize the term $a$, 
where $a~ >~ b~ >~ c$, the depth is infinite. 
To avoid this case,
there are two choices:
one is to use only terminating rules of which
every term in the condition is less than the left side of the rule
under a given reduction ordering; 
the other
is to limit the depth of each rewriting to a reasonable number.
Both criteria work well to guarantee the termination.
For pragmatic reasons, we prefer the second criterion.
In \ERRL, this number can be set by the command 
{\em option/brake/depth}; the default depth is 3.

\subsection{Strategy for Processing Next Equation}

There are three sources of equations: 
\begin{enumerate}
\item the user (input), 
\item critical pairs and 
\item deleted rules.
\end{enumerate}
  
The user's equations (except logical formulas) are processed in the
order in which they were input (queue).  The equations (except logical
formulas) obtained from critical pairs, are processed as soon as a
critical pair is generated. Rules are added (if necessary) soon after
the critical pairs are computed and processed.  As critical pairs are
processed immediately, many rules are added during the superposition
process. If these newly added rules delete some other rules, the
deleted rules can be processed in three different ways. This is
discussed in a later subsection.

There are three options to organize the processing of logical formulas
(no matter how they are generated): (1) stack, (2) queue (default),
(3) sorted list.  The command to set option is {\em option/fopc/list}.\\ \\
\tt
RRL-> {\em option fopc}\\ \\
Type break-ass, bound, restrict, list, match, paramodulation, quit, help --> {\em help}
\begin{verbatim}
break-ass        - Specify how long assertions should be broken 
bound            - Give a bound on number of rules made from critical pairs
restrict         - Set restrictions on making superpositions
list             - Set strategy for organize formulas
match            - Set strategy for boolean matching
paramodulation   - Set flag for paramodulation
quit             - Quit
help             - Print this message.
\end{verbatim}
Type break-ass, bound, restrict, list, match, paramodulation, quit, help --> {\em list} \\
Organize assertions as stack (s) queue (q) or sorted list (l) ? \\
(current value: q) (s,q,l) \\ \\
\rm
The default is the {\em queue} organization.

Other parameters in the fopc option are explained later.

\subsection{Critical Pair Strategy: {\em option/critical}}

\RRL supports different options for processing critical pairs.\\ \\
\tt
RRL-> {\em option critical}\\
Type deleted, inside, organize, paramodulation, pick, quit, rulesize, \\
\hspace*{0.4in} 
     restrict, with, help --> {\em help}
\begin{verbatim}
deleted          - Set strategy for processing deleted rules
organize         - Set strategy for organizing marked rules
paramodulation   - Set flag for paramodulation
pick             - Set strategy for picking an unmarked rule
quit             - Quit
rulesize         - Specify a measure for the size of rules
restrict         - Set restrictions on making superpositions
with             - Set strategies for choosing rules to do critical pairs with
help             - Print this message.
\end{verbatim}
\rm
  
For picking a rule from
not previously chosen (unmarked) rules for generating critical pairs, three options
are currently supported: 
\begin{enumerate}
\item choose the earliest generated rule; 
\item choose the earliest generated smallest rule or
\item choose the latest generated smallest rule.  
\end{enumerate}
The default option is 2. 
After a rule is picked and its critical pairs are computed, it is marked.
The command to set the critical pair strategy is:
{\em option/critical/pick}.
  
Having picked a rule based on a selected option, 
there are again three options to chose the 
other rules to compute critical pairs with the picked rule: 
\begin{enumerate}
\item choose all earlier generated rules (compared to the picked rule);
\item choose the rules previously picked for computing critical pairs (default for generating complete systems for equational
theories);
\item choose all rules except that those already
computed critical pairs with the currently picked rule (default
for theorem proving in predicate calculus).
\end{enumerate}
The command to select this option is: {\em option/critical/with}.
  
There are three ways to organize previously picked (marked) rules: 
\begin{enumerate}
\item stack, 
\item queue, 
\item sorted list (default). 
\end{enumerate}
The command to set this option is {\em option/critical/organize}. 

There are three options available to process equations converted from
deleted rules:

\begin{enumerate}
\item Process it as soon as a rule is deleted;
\item Process all deleted rules only after a critical pair is computed
and is processed (default option);
\item Process them after quitting the superposition process.
\end{enumerate}
The command to set these options is {\em option/critical/deleted}. 
For example,\\ \\
\tt
RRL-> {\em option critical}\\
Type deleted, inside, organize, paramodulation, pick, quit, rulesize, \\
\hspace*{0.4in} 
     restrict, with, help --> {\em deleted} \\
Process deleted rule at once (1), or all deleted rules together (2), or \\
after the superposition process (3) \\
(current value: 2) (1/2/3) {\em 3} \\
\rm
\\ If a rule is deleted outside of the superposition process, then 
the deleted rule will be processed immediately.

\subsection{Restricting Superpositions}

There are three options for restricting superpositions
between rules corresponding
to logical formulas for first-order
predicate calculus theorem proving. With these restrictions,
the theorem proving methods for first-order predicate calculus are
incomplete.
\begin{enumerate}
\item don't compute critical pairs with itself (idempotency);
\item compute critical pairs with unit rules (clause)
only; this is similar to unit resolution
strategy which is not refutationally complete;
\item compute critical pairs with input rules (clause)
only; this is similar to input resolution
strategy which is not refutationally complete.
\end{enumerate}

These options are included for the sake of efficiency.
This is done using {\em option/fopc/restrict}.\\ \\
\tt
RRL-> {\em option critical restrict} \\
Superposing with any rule (1), without itself (2), with unit rules (3), \\
\hspace*{0.4in} 
    or with input rules (4) ? \\
(current value: 1) (1,2,3,4) 
\rm

\section{Speeding Up the Completion Procedure: \em option/fastkb}

In order to speed-up the completion procedure for the associative-commutative
case, two different critical pair criteria have been implemented. In addition,
when a theory includes associative-commutative function symbols $+$ and $*$
such that $*$ distributes over $+$, then terms can be represented
more efficiently in sum of products form with these properties
of $+$ and $*$ built into the term structure.
  
\subsection{Blocking:\em option/fastkb/blocking}

In the completion procedure, certain critical pairs which are obtained
by substitutions that are reducible need not be considered as they
can be proved to be redundant \cite{KMN84}. Identifying such critical
pairs does not turn out to be efficient for examples not involving
associative-commutative functions. However, in the presence of any
associative-commutative functions, not considering such critical pairs
turn out to be quite efficient as they are many such critical pairs.
As a result, the default value for the option {\em blocking} is {\em
on} in the presence of associative-commutative functions.  However, it
can be turned off by the user if desired. Since there is
hardly any gain in identifying unblocked critical pairs in the
absence of associative-commutative functions, the blocking option
is not supported when no function is associative-commutative.

\subsection{Symmetry Checking: \em option/fastkb/symmetry}

In the presence of associative-commutative functions,
some subterms of the left side of a rule may be ``symmetric''
(two terms are said to be {\em symmetric} in a rewrite rule
if one becomes the other after properly renaming variables in 
the rewrite rule).
In this case, similar to blocking, there is no need to superpose
other rules at these symmetric subterms more than once \cite{ZK89}.
Moreover, there is no need to process ``symmetric'' unifiers,
which are computed by an associative-commutative unification
algorithm, more than once. 
If no functions are associative-commutative, symmetry checking
is reduced to identity checking. That is, there is 
no need to superpose at identical subterms of a rule more than once.
\RRL does symmetry checking by default. 
The user may turn it off by the command {\em option/fastkb/symmetry}.

\subsection{Polynomial Processing: \em option/fastkb/polynomial}

When the axiom set contains the function $*$ and $+$,
where $+$ is an associative-commutative operator and $*$ satisfies
the distribution law over $+$, it is often inefficient to
simplify terms using the rewrite rules made from the distribution law.
\RRL provides an option to efficiently handle this case.\\
\\
{\tt RRL->} {\em option fastkb polynomial}
\begin{verbatim}
Set the polynomial flag on means that the following axioms will be built in:
             x + y == y + x
       (x + y) + z == x + (y + z)
       (x + y) * z == (x * z) + (y * z)
       z * (x + y) == (z * x) + (z * y)
       (x * y) * z == x * (y * z)
Do you want to set it ? (t,nil) 
\end{verbatim}

The function names $*$ and $+$ are fixed for the above axioms.
In order to turn the {\em polynomial} flag off, 
\RRL must be reinitialized. 

\section{Changing Defaults}

It is possible to change the default values of various options as well
as various flags if they do not suit the needs of the user.  This can
be done by the command {\em option}, as illustrated in the 
previous sections, or 
by creating a file {\em init-rrl.flag}. The initial values of
various flags different from their default values can be specified in
the file  {\em init-rrl.flag} 
which consists of a sequence of flag-name and its initial
value as follows:\\ \\ {\em delete-rule 1\\ pick-rule l\\
critical-with m\\}\\ The first line above, for example, sets the
processing-deleted-rule-strategy to 1, i.e., process deleted rules
only after computing a critical pair.  All the flags
and their default values are listed below
(use option to see their meaning).

\begin{tabular}{lcc}
{\bf Name} & {\bf Domain} & {\bf Default value} \\
{\tt trace} & \{1, 2, 3\} & 2 \\
{\tt pick-rule} & \{l, e, o\} & l \\
{\tt critical-with} & \{m, o, a\} & m \\
{\tt delete-rule} & \{1, 2, 3\} & 2 \\
{\tt mark-rule} & \{1, 2, 3\} & 3 \\
{\tt rule-size} & \{s, d\} & s \\
{\tt normalization} & \{o, i, g, m\} & o \\
{\tt blocking} & \{y, n\} & y \\
{\tt lrpo} & \{t, nil\} & t \\
{\tt manual} & \{t, nil\} & nil \\
{\tt new-operator} & \{1, 2\} & 1 \\
{\tt status} & \{1, 2\} & 2 \\
{\tt precedence} & \{1, 2\} & 1 \\
{\tt equal-arity} & \{y, n\} & y \\
{\tt resume-rule} & \{y, n\} & n \\
{\tt post-position} & \{1, 2, 3\} & 2 \\
{\tt max-position} & any integer & 5 \\
{\tt max-new-rule} & any integer & 50 \\
{\tt post-ass-list} & \{s, q, l\} & q \\
{\tt more-break} & \{y, n\} & n
\end{tabular}
\normalsize
\rm

\section{\RRL States:  {\em history, undo, clean, save, load}}

\RRL provides the ability for the user to backtrack to previous steps
when the completion procedure is invoked or
where any decision regarding the orientation of rules was made.  
In order to implement the backtracking mechanism, the
state of the completion procedure at every decision step is saved. The
state of \RRL consists of 
\begin{itemize}
\item the current sets of equations and rules,
\item whether a function is commutative or associative and commutative,
\item whether a function is a constructor, 
\item whether equations are being manually oriented, automatically
oriented or interactively oriented, and if not-manually oriented, the
precedence relation on functions, and status of every function, 
\item values of flags
indicating the strategy to be used for normalization, generating
critical pairs, processing critical pairs.
\end{itemize}

The following commands can be used to manipulate \RRL states:

\begin{description}
\item{\em undo} Whenever a choice is made by the user to extend
the precedence relation or the {\em kb} command is called, the state of
the \RRL is stored in a history. If the user subsequently finds that a
wrong choice was made at some stage, then it is possible to backtrack
by invoking {\em undo} until that stage, make a different choice and
proceed again. It is also possible to keep {\em undoing} previous
decisions until the original system is restored after which the {\em
undo} command has no effect.

\item{\em clean} To support the {\em undo} feature, \RRL keeps record of
the state of the system whenever the user gives a command that changes
the system state.  For large examples, storing this information takes
up considerable storage which makes the garbage collection operation
in LISP quite
expensive. It is possible for the user to wipe out the history by
invoking the {\em clean} command. As a result of clean, \RRL can become
faster; however the {\em undo} command does not have any effect.
If the user does not want to save any state into the history, 
he (or she) may set
the option by the command {\em option/history/no-history}.

\item{\em history} Save the current state of \RRL into the history.
This command is useful after {\em clean} is used.

\item{\em save} When the user has made considerable progress in working
on a problem using \RRL but had to interrupt his session because of
various reasons, \RRL provides a facility using which it is possible
for the user to resume at a later stage from the point of interruption
without losing any information.  The user can use {\em save} to save
the current state of \RRL into a file with the suffix {\tt .rrl}.

\item{\em load} Load a state of \RRL from a file created by {\em save}.
\end{description}
