\begin{thebibliography}{99}

\bibitem{AAR}{\em Association of Automated Reasoning} Newsletters, 1984-86.
 
\bibitem{Bachmair}Bachmair, L. (1987). {\em Proof methods for
equational theories.} Ph.D. Thesis, Dept. of Computer Science,
University of Illinois, Urbana.

\bibitem{BoyerMoore79}
Boyer, R.S. and Moore, J S., (1979) A computational logic.
{\em Academic Press}, New York.

\bibitem{Boyeretal}Boyer, R., Lusk, E., McCune, W., Overbeek, R.,
Stickel, M., Wos, L. (1986). Set theory in first-order logic: clauses
for Goedel's Axioms. {\em J. of Automated Reasoning} 2, 3, 287-327.

\bibitem{Bundy}Bundy, A. (1985). {\em The computer modelling of
mathematical reasoning}. Academic Press, New York.

\bibitem{Carroll}Carroll, L. (1958). {\em Symbolic logic and game of
logic}.  Dover Publications, New York.

\bibitem{Dershowitz82}Dershowitz, N. (1982). Orderings for term
rewriting systems. {\em Theoretical Computer Science} 17, 279-301.

\bibitem{Dershowitz85}Dershowitz, N. (1985). Computing with
rewrite systems. {\em Information and Control} 64, 122-157.

\bibitem{Dershowitz87}Dershowitz, N. (1987). Termination of rewriting.
{\em J. of Symbolic Computation}.

\bibitem{DetlefsForgaard}Detlefs, D., and Forgaard, R. (1985). A
Procedure for automatically proving the termination of a set of
rewrite rules. 
In \cite{RTA85} 
255-270.

\bibitem{ForgaardGuttag}Forgaard, R., and Guttag, J.V. (1984). REVE:
A term rewriting system generator with failure-resistant Knuth-Bendix.
In Reference \cite{Guttagetal84}.

\bibitem{Fortenbacher}Fortenbacher, A. (1985). An algebraic approach
to unification under associativity and commutativity. 
In \cite{RTA85}.

\bibitem{GuttagHorning}Guttag, J., and Horning, J.J. (1978). The
algebraic specification of abstract data types. {\em Acta Informatica}
10 (1), 27-52.

\bibitem{Guttagetal84}Guttag, J.V., Kapur, D., Musser, D.R. (eds.)
(1984).  Proceedings of an {\em NSF Workshop on the Rewrite Rule
Laboratory} Sept. 6-9 Sept. 1983, General Electric Research and
Development Center Report 84GEN008.

\bibitem{Hsiang85}Hsiang, J. (1985). Refutational theorem proving
using term rewriting systems. {\em Artificial Intelligence} Journal,
25, 255-300.

\bibitem{HR86} Hsiang, J., and Rusinowitch, M. (1986). A new method for 
establishing refutational completeness in theorem proving.
{\em Proc. 8th Conf. on Automated Deduction},
LNCS No. 230, Springer Verlag, 141-152.

\bibitem{Huet80}Huet, G. (1980). Confluent reductions: abstract
properties and applications to term rewriting systems. {\em J. ACM}
27, 4, 797-821.

\bibitem{HuetHullot}Huet, G., and Hullot, J.M. (1980). Proofs by
induction in equational theories with constructors.  {\em 21st IEEE
Symposium on Foundations of Computer Science}, 96-107.

\bibitem{JK86}Jouannaud, J., and Kounalis, E. (1986).
Automatic proofs by induction in equational theories without
constructors.  Proc. of {\em Symposium on Logic in Computer Science},
358-366.

\bibitem{JAR}{\em Journal of Automated Reasoning}, Problems Corner.
 
\bibitem{JSC} {\em Journal of Symbolic Computation}, 3, 1987.

\bibitem{KaminLevy}Kamin, S., and Levy, J-J. (1980). Attempts for
generalizing the recursive path ordering. Unpublished Manuscript,
INRIA.

\bibitem{KapurMusser84}Kapur, D., and Musser, D.R. (1984). Proof by
consistency.  In Reference \cite{Guttagetal84}, 245-267.
Also in {\em Artificial Intelligence} Journal, 

\bibitem{KMN84}Kapur, D., Musser, D.R., and Narendran, P. (1988). Only
prime superpositions need be considered for the Knuth-Bendix
completion procedure.  
{\em Journal of Symbolic Computation} Vol. 4.
 
\bibitem{KN85}Kapur, D., and Narendran, P. (1985).
An equational approach to theorem proving in first-order
predicate calculus. Proc. of {\em 8th IJCAI}, Los Angeles, Calif.
An expanded version appeared as a GE Technical Report, 84CRD232.

\bibitem{KN87} Kapur, D., and Narendran, P. (1987). Matching, Unification and
Complexity. {\em SIGSAM Bulletin}.

\bibitem{KNZ85}Kapur, D., Narendran, P., and Zhang, H (1987). On
sufficient completeness and related properties of term rewriting
systems. 
{\em Acta Informatica,} Vol. 24, Fasc. 4, 395-416.

\bibitem{KNZ86}Kapur, D., Narendran, P., and Zhang, H. (1986). Proof
by induction using test sets. {\em 8th Intl. Conf. on Automated
Deduction}, Lecture Notes in Computer Science, 230, Springer Verlag, 
New York.

\bibitem{KapurSiva}Kapur, D. and Sivakumar, G. (1984) Architecture of
and experiments with RRL, a Rewrite Rule Laboratory. In: Reference
\cite{Guttagetal84}, 33-56.

\bibitem{KSZ86}Kapur, D., Sivakumar, G., and Zhang, H. (1986). RRL: A
Rewrite Rule Laboratory. {\em 8th Intl. Conf. on Automated Deduction},
Lecture Notes in Computer Science, 230, Springer Verlag, New York.

\bibitem{KZ881} Kapur, D., and Zhang, H. (1988).
{\em RRL: A Rewrite Rule Laboratory}.  Proc. of {\em Ninth
International Conference on Automated Deduction} (CADE-9), Argonne,
IL, May 1988.

\bibitem{KZ882} Kapur, D., and Zhang, H. (1988). Proving equivalence of 
different axiomatizations of free groups.  {\em J. of Automated
Reasoning} 4, 3, 331-352.

\bibitem{KnuthBendix}Knuth, D.E. and Bendix, P.B. (1970). Simple word
problems in universal algebras.  In: {\em Computational Problems in
Abstract Algebras.} (ed. J.  Leech), Pergamon Press, 263-297.

\bibitem{LB77}Lankford, D.S., and Ballantyne, A.M.
(1977). Decision procedures for simple equational theories with
commutative-associative axioms: complete sets of
commutative-associative reductions. Automatic Theorem Proving Project,
Dept. of Math. and Computer Science, University of Texas, Austin,
Texas, Report ATP-39.

\bibitem{Lescanne}Lescanne, P. (1983). Computer experiments with the
REVE term rewriting system generator. Proc. of {\em 10th Principles of
Programming Languages (POPL)} Conference.

\bibitem{Musser80}Musser, D.R. (1980). On proving inductive
properties of abstract data types. Proc. {\em 7th Principles of
Programming Languages (POPL)}.

\bibitem{MusserKapur}Musser, D.R., and Kapur, D. (1982). Rewrite rule
theory and abstract data type analysis. {\em Computer Algebra,
EUROCAM, 1982} (ed. Calmet), Lecture Notes in Computer Science 144,
Springer Verlag, 77-90.

\bibitem{NS881} Narendran, P., and Stillman, J. (1988). 
Hardware verification in the 
Interactive VHDL Workstation. In: {\em VLSI Specification, Verification 
and Synthesis } (eds. G. Birtwistle and P.A. Subrahmanyam), Kluwer 
Academic Publishers, 217-235.

\bibitem{NS882} Narendran, P., and Stillman, J. (1987).
Formal verification of the Sobel image processing chip. Unpublished
Manuscript, General Electric Corporate Research and Development,
Schenectady, NY, November 1987. Submitted for publication.

\bibitem{PS81}Peterson, G.L., and Stickel, M.E. (1981).
Complete sets of reductions for some equational theories. {\em J.
ACM}, 28, 2, 233-264.

\bibitem{RemyZhang} Remy, J.L., and Zhang, H. (1984). 
REVEUR4: A system for validating conditional algebraic specifications
of abstract data types. Proc. of the {\em 6th ECAI,} Pisa, 563-572.

\bibitem{RTA85}Proc. of the {\em First International Conference on
Rewriting Techniques and Applications (RTA-85)}, Dijon, France,
Lecture Notes in Computer Science, 202, Springer Verlag, New York.

\bibitem{Smullyan1}Smullyan, R. (1985). {\em To mock a mocking bird
and other logical puzzles}. Knopf Publications, New York.

\bibitem{Smullyan2}Smullyan, R. (1982). {\em The lady or the tiger?
and other logical puzzles}. Knopf Publications, New York.

\bibitem{Stickel84}
Stickel, M.E. (1984). A case study of theorem proving by the
Knuth-Bendix method: discovering that $x^3 = x$ implies ring
commutativity. Proc. of {\em 7th Conf. on Automated Deduction},
Springer-Verlag LNCS 170, pp. 248-258.

\bibitem{Stickel}Stickel, M. (1986). Schubert's steamroller problem:
formulations and solutions. {\em J. of Automated Reasoning} 2, 1,
89-101.

\bibitem{Walther}Walther, C. (1985). A mechanical solution to
Schubert's steamroller problem by many-sorted resolution. {\em
Artificial Intelligence} Journal 26, 217-224.

\bibitem{Zhang88} 
Zhang, H., (1988) {\em Reduction, Superposition and Induction: Automated
Reasoning in an Equational Logic,} Ph.D. Thesis, Department of
Computer Science, Rensselaer Polytechnic Institute, Troy, NY.
Also {\em Technical Report 88-06}, Dept. of Computer Science,
University of Iowa.

\bibitem{ZK88} Zhang, H., and Kapur, D. (1988). First-order
theorem proving using conditional rewriting. 
Proc. of {\em Ninth International
Conference on Automated Deduction} (CADE-9), Argonne, IL, May 1988.
Springer-Verlag LICS 310.

\bibitem{ZKM88} Zhang, H., Kapur, D., and Krishnamoorthy,
M.S.  (1988). A mechanizable induction principle for equational
specifications. 
Proc. of {\em Ninth International
Conference on Automated Deduction} (CADE-9), Argonne, IL, May 1988.
Springer-Verlag LICS 310.

\bibitem{ZK89} Zhang, H., and Kapur, D. (1989). Consider
only general superpositions in completion procedures. {\em Proc.  of
Third International Conf. on Rewrite Techniques and Applications},
Chapel Hill, NC.

\bibitem{ZhangRemy}Zhang, H., and Remy, J.L. (1985). Contextual
rewriting. In \cite{RTA85}, 46-62.

\end{thebibliography}
