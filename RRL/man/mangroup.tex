\chapter{Generating a canonical system for free groups}
\normalsize
\rm
The following equational axiomatization of free groups is a pet
example in the literature on rewriting techniques, perhaps because the
seminal paper by Knuth and Bendix used this example.  The goal is to
start with three equation axiomatization of free groups and generate a
complete rewriting system for free groups using which any equation
expressed only in terms of the group operations can be decided to be
in the first-order equational theory of free groups. After the
complete rewriting system is generated, it is shown near the end of
the transcript how this system can be used to prove and disprove
equations about free groups.

For generating a complete system for an axiomatization using the
interactive mode for orienting rules from equations, the following
steps can be followed: 
\begin{enumerate}
\item Give an axiomatization as input.
\item (optional) Specify precedence relation on function symbols
using the command {\em oper precedence}.
\item Invoke {\em kb}.
\item Give precedence relation incrementally when equations cannot be
oriented.
\item (optional) Subsequently, if you wish to change the 
precedence relation, {\em undo} to the step where
the precedence relation that must be changed was given.
\end{enumerate}
If the completion procedure stops, a complete system is generated
which can be used to prove or disprove equations using
the {\em prove} command. If it appears that the completion procedure
may go forever, then the computation can be interrupted
using {\em $uparrow$-C} or {\em abort}.

\small 
\tt 
\begin{verbatim}
****************************************************************************
****                                                                    ****
****               WELCOME TO REWRITE RULE LABORATORY (RRL)             ****
****                                                                    ****
****************************************************************************

Type Add, Akb, Auto, Break, Clean, Delete, Grammar, History, Init, Kb, List,
     Load, Log, Makerule, Narrow, Norm, Option, Operator, Prove, Quit, Read,
     Refute, Save, Statics, Suffice, Undo, Unlog, Write or Help.
\end{verbatim}
RRL-> {\em read}\\
Type filename: {\em group}
\begin{verbatim}
Equations read in are:
 1. (e * x) == x  [user, 1]
 2. (i(x) * x) == e  [user, 2]
 3. ((x * y) * z) == (x * (y * z))  [user, 3]
New constant set is: { e }
Type Add, Akb, Auto, Break, Clean, Delete, Grammar, History, Init, Kb, List,
     Load, Log, Makerule, Narrow, Norm, Option, Operator, Prove, Quit, Read,
     Refute, Save, Statics, Suffice, Undo, Unlog, Write or Help.
\end{verbatim}
RRL-> {\em kb}
\begin{verbatim}
----- Step 1 -----
Adding rule:   [1] (e * x) ---> x  [user, 1]
Trying to orient equation: 
  (i(x) * x) == e  [user, 2]
  To prove:  (i(x) * x)  >  e
  Here are some precedence suggestions:
       1.  * > e
       2.  i > e
Either type a list of numbers or
Type Abort, Display, Drop, Equiv, LR, MakeEq, Operator, Postpone, Quit, RL,
     Status, Superpose, Twoway, Undo or Help.
\end{verbatim}
RRL>KB> {\em 1 2}
\begin{verbatim}
----- Step 2 -----
Adding rule:   [2] (i(x) * x) ---> e  [user, 2]
Trying to orient equation: 
  ((x * y) * z) == (x * (y * z))  [user, 3]
  To prove:  ((x * y) * z)  >  (x * (y * z))
        or:  ((x * y) * z)  <  (x * (y * z))
I have no precedence suggestions.  
Try doing Equiv or Status.
Type Abort, Display, Drop, Equiv, LR, MakeEq, Operator, Postpone, Quit, RL,
     Status, Superpose, Twoway, Undo or Help.
\end{verbatim}
RRL>KB> {\em s}\\
Type operator you wish to set status for: {\em *}\\
Type lr, rl, help --> {\em lr}
\begin{verbatim}
Operator, *, given status: lr
----- Step 3 -----
Adding rule:   [3] ((x * y) * z) ---> (x * (y * z))  [user, 3]
Computing critical pairs with: 
    [1] (e * x) ---> x  [user, 1]
Computing critical pairs with: 
    [2] (i(x) * x) ---> e  [user, 2]
Computing critical pairs with: 
    [3] ((x * y) * z) ---> (x * (y * z))  [user, 3]
Adding rule:   [4] (i(y) * (y * z)) ---> z  [2, 3]
Computing critical pairs with: 
    [4] (i(y) * (y * z)) ---> z  [2, 3]
Adding rule:   [5] (i(e) * z) ---> z  [1, 4]
Adding rule:   [6] (i(i(z)) * e) ---> z  [2, 4]
Adding rule:   [7] (i((x * y)) * (x * (y * z))) ---> z  [3, 4]
Adding rule:   [8] (i(i(y)) * z) ---> (y * z)  [4, 4]
  Deleting rule:  [6] (i(i(z)) * e) ---> z  [2, 4]
Adding rule:   [9] (z * e) ---> z  [deleted, 6]
Computing critical pairs with: 
    [9] (z * e) ---> z  [deleted, 6]
Adding rule:  [10] i(e) ---> e  [2, 9]
  Deleting rule:  [5] (i(e) * z) ---> z  [1, 4]
Computing critical pairs with: 
   [10] i(e) ---> e  [2, 9]
Computing critical pairs with: 
    [8] (i(i(y)) * z) ---> (y * z)  [4, 4]
Adding rule:  [11] i(i(y)) ---> y  [9, 8]
  Deleting rule:  [8] (i(i(y)) * z) ---> (y * z)  [4, 4]
Computing critical pairs with: 
   [11] i(i(y)) ---> y  [9, 8]
Adding rule:  [12] (y * i(y)) ---> e  [11, 2]
Adding rule:  [13] (y * (i(y) * z)) ---> z  [11, 4]
Computing critical pairs with: 
   [12] (y * i(y)) ---> e  [11, 2]
Adding rule:  [14] (x * (y * i((x * y)))) ---> e  [3, 12]
Computing critical pairs with: 
   [13] (y * (i(y) * z)) ---> z  [11, 4]
Adding rule:  [15] (x * (y * (i((x * y)) * z))) ---> z  [3, 13]
Computing critical pairs with: 
   [14] (x * (y * i((x * y)))) ---> e  [3, 12]
Adding rule:  [16] (x * (y * (y1 * i((x * (y * y1)))))) ---> e  [3, 14]
Adding rule:  [17] (y * i((y1 * y))) ---> i(y1)  [14, 4]
  Deleting rule: [14] (x * (y * i((x * y)))) ---> e  [3, 12]
Computing critical pairs with: 
   [17] (y * i((y1 * y))) ---> i(y1)  [14, 4]
Adding rule:  [18] (x * (y * i((y1 * (x * y))))) ---> i(y1)  [3, 17]
  Deleting rule: [16] (x * (y * (y1 * i((x * (y * y1)))))) ---> e  [3, 14]
Adding rule:  [19] (y * i((x * (y1 * y)))) ---> i((x * y1))  [3, 17]
  Deleting rule: [18] (x * (y * i((y1 * (x * y))))) ---> i(y1)  [3, 17]
Adding rule:  [20] (x * (i((y * x)) * z)) ---> (i(y) * z)  [17, 3]
  Deleting rule: [15] (x * (y * (i((x * y)) * z))) ---> z  [3, 13]
Trying to orient equation: 
  (i(y) * i(y1)) == i((y1 * y))  [17, 4]
  To prove:  (i(y) * i(y1))  >  i((y1 * y))
        or:  (i(y) * i(y1))  <  i((y1 * y))
  Here are some precedence suggestions:
       1.  * > i                     2.  i > *
Either type a list of numbers or
Type Abort, Display, Drop, Equiv, LR, MakeEq, Operator, Postpone, Quit, RL,
     Status, Superpose, Twoway, Undo or Help.
\end{verbatim}
RRL>KB> {\em 2}
\begin{verbatim}
----- Step 4 -----
Adding rule:  [21] i((y * y1)) ---> (i(y1) * i(y))  [17, 4]
  Deleting rule:  [7] (i((x * y)) * (x * (y * z))) ---> z  [3, 4]  
  Deleting rule: [17] (y * i((y1 * y))) ---> i(y1)  [14, 4]
  Deleting rule: [19] (y * i((x * (y1 * y)))) ---> i((x * y1))  [3, 17]
  Deleting rule: [20] (x * (i((y * x)) * z)) ---> (i(y) * z)  [17, 3]
Computing critical pairs with: 
   [21] i((y * y1)) ---> (i(y1) * i(y))  [17, 4]
Your system is canonical.
  [1] (e * x) ---> x  [user, 1]
  [2] (i(x) * x) ---> e  [user, 2]
  [3] ((x * y) * z) ---> (x * (y * z))  [user, 3]
  [4] (i(y) * (y * z)) ---> z  [2, 3]
  [9] (z * e) ---> z  [deleted, 6]
 [10] i(e) ---> e  [2, 9]
 [11] i(i(y)) ---> y  [9, 8]
 [12] (y * i(y)) ---> e  [11, 2]
 [13] (y * (i(y) * z)) ---> z  [11, 4]
 [21] i((y * y1)) ---> (i(y1) * i(y))  [17, 4]
Processor time used                = 4.67 sec
Number of rules generated          = 21
Number of critical pairs           = 73
Time spent in normalization        = 1.27 sec (27.1 percent of time)
Time spent while adding rules      = 0.9 sec (19.3 percent of time)
  (keeping rule set reduced)
Time spent in unification          = 0.53 sec (11.4 percent of time)
Time spent in ordering             = 0.45 sec (9.6 percent of time)
Total processor time used (include 'undo' action) = 4.77 sec

Type Add, Akb, Auto, Break, Clean, Delete, Grammar, History, Init, Kb, List,
     Load, Log, Makerule, Narrow, Norm, Option, Operator, Prove, Quit, Read,
     Refute, Save, Statics, Suffice, Undo, Unlog, Write or Help.
\end{verbatim}
RRL-> {\em prove} 
\begin{verbatim}
Type equation to prove in the format:  L == R (if C) 
Enter a ']' to exit when no equation is given.
\end{verbatim}
{\em i(i(y1) * i(y)) == y * y1} \\
\tt
Yes, it is equational theorem.
\rm
\normalsize
