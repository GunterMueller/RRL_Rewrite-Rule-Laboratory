\chapter{Commands}

\section{Top Level: {\em rrl}}
When the user types {\em rrl} on a machine where \RRL has been
installed, the user enters the top level and gets the following
message:

\begin{verbatim}
****************************************************************************
****                                                                    ****
****               WELCOME TO REWRITE RULE LABORATORY (RRL)             ****
****                                                                    ****
****************************************************************************

Type Add, Akb, Auto, Break, Clean, Delete, Grammar, History, Init, Kb, List,
     Load, Log, Makerule, Narrow, Norm, Option, Operator, Prove, Quit, Read,
     Refute, Save, Statics, Suffice, Undo, Unlog, Write or Help.
RRL-> 
\end{verbatim}  
At that stage, \RRL is ready to accept any of the commands listed
above.  The top-level prompt by \RRL is {\tt RRL->}.  We group
commands by their functionality and describe each command below. A
list of commands with a single sentence description is given in
Appendix A.
  
In this manual, we use {\em /} to separate successive commands. Each
command has its sub-sequence starting with its initial letter as its
abbreviated name insofar as the subsequence disambiguates the command.
For example, {\em r} is the same as {\em read}.  For clarity, we shall
mostly use the full names of the commands in this manual.
In the following, \RRL prompts are in {\tt typewriter} font while the
user's response is {\em emphasized}.

\section{Input and Output: {\em add, read, write, delete, list, init}}

The input to \RRL is a set of equations or formulas in first-order logic.
\begin{description} 
   \item{\em add} reads input from the keyboard.
   \item{\em read} reads input from a file prepared by the user. 
   \item{\em write} writes the equations and/or rules
at that stage into a file specified by the
user. 
   \item{\em list} displays the equations and rules in the
system at that stage.
   \item {\em delete} deletes specified rules or equations from the current
system. One parameter ({\em e/r}) and a list of numbers are needed to tell
\RRL to delete the specified equations ({\em e}) or rules ({\em r}).
Note that the number of each rule is fixed while the number of an equation
is floating. That is, after deleting the first equation, then the second
equation (if any) becomes the number one.
   \item {\em init} Cleans out all rules and equations. All global parameters
and flags are
initialized. All flags are set to initial values.  \RRL is ready to be
executed on a new input.
\end{description}

For the command {\em read}, in a file name {\em myfile.eqn}, 
it is sufficient to specify {\em myfile}. 
When the {\em write} command is typed, 
the user is prompted for whether the whole
system is to be saved, or a component of it, as illustrated below: 
\\ \\
{\tt Type quit, equations-only, rules-only, all, kb-state, help --> {\em help}}
\begin{verbatim}
quit             - Quit without saving anything.
equations-only   - Save only equations.
rules-only       - Save only rules.
all              - Save both equations and rules.
kb-status        - Save KB statistics in a file with the suffix `.out'.
help             - Print this message.  
\end{verbatim}
The user must choose one from the above list.
  
The syntax of equations or formulas is given in Appendix B.
\RRL distinguishes among five categories of tokens (or words):
\begin{description}
 \item{\bf Meta tokens:}  {\tt \{ ,  (  )  [ ]  ==  ---> := if ; \} };
 \item{\bf Logic operators:} {\tt true, false, not, and, or, xor 
\rm (exclusive-or), \tt -> \rm (implication), \\
\tt equ \rm (equivalence), \tt all, exist\rm };
 \item{\bf Variables:}  any alphanumeric strings 
beginning with {\tt u, v, w, x, y, {\rm or} z};
 \item{\bf Predicates:} any printable strings except the above and numbers.
The symbol {\tt eq} stands for a pre-defined
equality predicate.
 \item{\bf Operators:} same as predicates and in addition, numbers as constants.
\end{description}
  
The print names of the operators {\tt and} and {\tt xor} are {\tt \&}
and {\tt +}, respectively. The symbol {\tt \&}
can be used in the input for {\tt and} but {\tt +}
cannot be used for {\tt xor}. We now give some
examples. \\ \\
{\tt 
RRL-> {\em add} \\
Type your equations, enter a ']' when done.\\
\em x * 0 == x\\
x * (1 + y) == x + (x * y)\\
x / y == x * i(y) if not(eq(y, 0))\\
iszero(0)\\
iszero(x * y) equ (iszero(x) or iszero(y))\\
]} \\ \\
The first two equations in the above input are unconditional
equations, {\tt x} is a variable, {\tt *} is a binary infix operator,
{\tt ==} separates the left side of an equation from the
right side.  The third is a conditional equation -- {\tt if}
separates the condition from the equation. The last two are formulas
where {\em iszero} is a predicate.
  
Formulas are abbreviated forms of equations. Each of
the following two
equations, for example, is equivalent to the last formula 
in the above example.
\begin{center}
\em iszero(x * y) equ (iszero(x) or iszero(y)) == true\\
iszero(x * y) == (iszero(x) or iszero(y))
\end{center}
  
Every operator and predicate are assumed to be of a fixed arity.  A
binary operator in an expression (term) can be written as infix or prefix. 
The following precedence relation among predicate or logical operators is 
assumed: 
\begin{verbatim}
        predicates > not > and > -> > { all, exists } > equ > { or, xor }
\end{verbatim}
For example, {\tt (p and q or r)} is parsed as {\tt ((p and q) or r)}.
When the parentheses between infix operators are omitted, the parser
assumes them to be right-associative. For example, {\tt x + y + z + u}
is parsed as {\tt (x + (y + (z + u)))}.
  
An equation can take more than one line to write. The last token in an
unfinished line must be an infix operator, 
the character {\tt (} or an {\tt ,}.
  
\RRL appends a
message at the end of each equation or rule, giving the source of
that equation or rule. For example, {\tt [user, 1]} means that
equation (or rule) is the first one given by the user. ${\tt [deleted, 10]}$
means that equation (or rule) is made from rule 10 which
got deleted because its left side could be rewritten
by some other rule. ${\tt
{[3, 4]}}$ means that equation (or rule) is made from a critical pair of
the rules 3 and 4.

{\em add} and {\em read} commands can be also used to specify
the arity of an operator. For example, to specify that $0$ is a constant
of sort $num$, $suc$ is a unary function of sort $num$, 
$+$ is a binary function on $num$, and $odd$
is a predicate defined on $num$, we may declare them by \\ 
\\ 
{\tt 
RRL-> {\em add}\\
Type your equations, enter a ']' when done.\\
\em {[0: num]} \\
{[suc: num -$>$ num]} \\
{[+ : num, num -$>$ num]} \\
{[odd : num -$>$ bool]} \\
]} \\ \\
Arity (or sort) declarations can be mixed with equations in any order.
A good practice is to declare an operator first, then followed by
its definition. $bool$ and $univ$ are two pre-defined sort names.
If no declaration is given for a function, then by default
that function is of sort $univ$, which is a super-sort containing any other
sort. Any predicate is assumed to be of sort $bool$.

\section{Specifying Properties of Operators: {\em operator}}

In addition to axioms characterizing operators,
some properties of operators can also be
specified using commands in \ERRL. \\ \\
\tt 
RRL-> {\em operator} \\
Type order, rulesize, display, constructor, commutative, acoperator, \\
\hspace*{0.4in}
     transitive, divisible, equivalence, precedence, quit, status, weight, \\
\hspace*{0.4in}
     help --> \em help
\begin{verbatim}
order         - Choose the ordering to be used for ensuring termination.
rulesize      - Choose a measure for the size of rules.
display       - Display precedence and status of operators for ordering.
constructor   - Declare operators to be constructors.
commutative   - Declare operators to be commutative.
acoperator    - Declare operators to be associative and commutative.
transitive    - Declare operators to be transitive.
divisible     - Declare operators to be left and right divisible.
equivalence   - Specify operators whose precedence is equivalent for ordering.
precedence    - Specify precedence on operators for ordering.
quit          - Quit.
status        - Specify the status of an operator.
weight        - Set the weight of operators (used for ordering and rulesize?).
help          - Print this message.
\end{verbatim}
\rm
For example,
the commutativity property of an operator can be
specified using the command {\em operator/commutative} as follows: \\ \\
\tt
RRL-> {\em operator}\\
Type order, rulesize, display, constructor, commutative, acoperator, \\
\hspace*{0.4in}
     transitive, divisible, equivalence, precedence, quit, status, weight, \\
\hspace*{0.4in}
     help --> {\em commutative} \\
Type operators you wish to make commutative: {\em f}\\
The operator 'f' is commutative now.\\ \\
\rm
Similarly, an operator with the associativity and commutativity can be
specified using the command {\em operator/acoperator}. 
A transitive relation
(predicate) can be specified using the command {\em operator/transitive}.
  
Algorithms for checking the sufficient completeness property of a
specification as well as for proving equations by induction require that
certain functions in an input theory be distinguished to be
constructors based on their semantics. Constructors are functions that
are solely used to construct the values in a universe of discourse.
An operator can be specified to be a constructor using the command
{\em operator/constructor}.

As discussed later, \RRL has an algorithm based on lexicographic
recursive path ordering for determining the orientation of equations
to terminating rules. This algorithm uses precedence relation on the
function symbols in the formulas.  The {\em operator} command is used
to specify the precedence information, such as equivalence or
inequivalence among operators, precedence on operators as well as
their status, i.e., the order in which their arguments must be
compared. \\ \\
\tt
RRL-> {\em operator}\\
Type order, rulesize, display, constructor, commutative, acoperator, \\
\hspace*{0.4in}
     transitive, divisible, equivalence, precedence, quit, status, weight, \\
\hspace*{0.4in}
     help --> {\em precedence} \\
Type operators in decreasing precedence.\\
\hspace*{0.4in} (eg. 'i * e' to set i > * > e) \\
--> {\em i * e}\\
Precedence relation, i > *, is added.\\
Precedence relation, * > e, is added.\\
\hspace*{0.4in} ...\\
RRL-> {\em operator}\\
Type order, rulesize, display, constructor, commutative, acoperator, \\
\hspace*{0.4in}
     transitive, divisible, equivalence, precedence, quit, status, weight, \\
\hspace*{0.4in}
     help --> {\em status} \\
Type operator you wish to set status for: {\em *}\\
Type lr, rl, help --> {\em lr}\\
Operator, *, given status: lr \\ \\
\rm
The precedence information specified so far can be
displayed using the {\em display} command within
{\em operator} command.\\ \\
\tt
RRL-> {\em operator}\\
Type order, rulesize, display, constructor, commutative, acoperator, \\
\hspace*{0.4in}
     transitive, divisible, equivalence, precedence, quit, status, weight, \\
\hspace*{0.4in}
     help --> {\em display}
\begin{verbatim}
There are no equivalent operators.
Precedence relation now is: 
   i > * e 
   * > e 
Operators with status are:
   * with left-to-right status.
\end{verbatim}
\rm
In case of logical formulas, rules are made from them assuming a total
lexicographic order on predicate symbols unless stated differently. So,
if the user wishes to impose a different precedence relation on
predicate symbols, it must be stated using the {\em
operator/precedence} command before invoking the completion procedure.

In addition to specifying these properties of operators, the {\em
operator/order/m} command is used to specify whether equations are to be
oriented manually.  In that case, it is the user's responsibility to
orient equations into rules so that the rewriting process terminates.

