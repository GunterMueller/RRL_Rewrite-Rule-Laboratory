\section{Narrowing: {\em narrow}}

\RRL also supports the narrowing method of solving an equation in a
given theory. In this method, in contrast to proof by normalization,
two sides of the equation being solved are {\em narrowed} (unified
with the left sides of rules) to look for a substitution of
variables in the equation such that the result of applying the
substitution on the equation is another equation which is a theorem in
the given theory.

It is often helpful at first to generate a complete system for the
given theory, especially when many equations are intended to be solved
in the theory, but this is not necessary. In case, there does not
exist a complete system for the given theory or the user does not wish
to generate a complete system because of considerable overhead
involved in generating a complete system, rules can be generated from
the formulas in the theory. These rules are then used to solve the
equation using the linear completion procedure proposed by Dershowitz
\cite{Dershowitz85}.

The method can be used to implement logic programming in the rewriting
paradigm. See an example in Appendix I.  We are not convinced of the
usefulness of this method of solving equations because of its
inefficiency. As a result, we have not experimented with the method or
provided strategies to support different variations of the method.

The {\em narrow} command in \RRL
supports the narrowing concept.  Before the {\em narrow} command is
invoked, formulas in a theory should be made into rules which may or
may not be completed using the completion procedure. It is a good idea
to complete a theory, if possible, especially in case many equations
needs to be solved.  Completion in this case serves the role of
compilation.

Given a set of rules, the {\em narrow} command expects an equation as
the input. Then, the linear completion procedure is invoked which
gives a set of solutions to the equation. Additional solutions, if
any, can be obtained by invoking the {\em narrow} command again and
{\em continue} command. The user should be warned that the linear
completion procedure often does not terminate.

Appendix I is an illustration of the {\em narrow} command on an
example from logic programming.

\section{Testing Sufficient Completeness: {\em suffice}} 

An equational axiomatization of functions can be checked for
sufficient completeness. At first, functions must be classified into
constructors and nonconstructors;
this can be done
using the {\em operator} command and within
it, by a command {\em constructor}.
A nonconstructor function is
sufficiently complete (or completely defined) if it gives a
constructor ground term (term without any variables built from
constructors) as a result for every input made of constructor ground
terms.

The {\em suffice} command is provided to check the sufficient
completeness property.  First, a complete rewriting system must be
generated for the equational axiomatization.  Two methods are
implemented in the current \RRL: one is based on the notion of {\em
ground-reducibility} \cite{JK86}, the other is based on the
concept of {\em test set} \cite{KNZ86}. Both methods need, as inputs,
a complete rewriting system and a subset of operations identified as
constructors.  If a nonconstructor function is not completely defined,
both the methods generates a finite set of schema on which the
function must be defined using additional axioms for it to be
completely defined. Either method can be chosen using
{\em option/prove} command and setting the value
to be {\em ground-reducibility} or {\em test-set}; the default
value is {\em test-set}.

The user should be warned that {\em suffice} command is not fully
integrated with the mechanisms in \RRL to handle
associative-commutative constructors. However, whenever \RRL claims
that a function is completely defined even in the presence of
associative-commutative constructors, barring bugs in \ERRL, the
function is completely defined. There do exist axiomatizations
involving associative-commutative constructors which completely define
functions, but \RRL may declare them to be not sufficiently complete.
  
The {\em suffice} command is illustrated in Appendix J.

\section {Theorem Proving: {\em kb, refute, prove}}

Four approaches to theorem proving
are supported in \ERRL; all of them are based on rewriting systems.
\begin{itemize}
\item proof by forward reasoning;
\item proof by refutation;
\item proof by consistency (inductionless induction);
\item proof by (explicit) induction.
\end{itemize}
For the last two approaches, 
in addition to the axioms, 
the user is required
to input 
the arity of each operator 
(including its domain sort and codomain sorts).
To choose which method to use, the command is {\em option/prove}:\\ 
\\
{\tt RRL->} {\em option prove}
\begin{verbatim}
Following methods are available:
    Deduction:
       (f) forward reasoning --- using completion
       (s) refutation        --- using boolean-ring method
       (c) refutation        --- using conditional rules
    Induction:
       (e) explicit          --- using the cover-set method
       (s) inductionless     --- using the test set method
       (g) inductionless     --- using the ground-reducibility method
Please make your choice: 
(current value: s) (f,c,e,s,g) 
\end{verbatim}
The default option is the {\em boolean-ring method} for 
deductive theorem proving
and the {\em test-set} method for inductive theorem proving.

\subsection{Proof by Forward Reasoning: \em prove} 

A special case of proof by forward reasoning is {\em proof by normalization},
which has been intergrated as the first step
into any other proof method.

As emphasized quite a bit above,
a complete rewriting system is a decision procedure
for the equational theory defined by the rewriting system.  An
equation is a theorem in the equational theory if and only if both sides of
the equation are normalized to the same term using the complete
rewriting system.  The command to do that is {\em prove}
which takes an equation as input.  Both unconditional or
conditional equation could be proved in this way.  
If the proof by normalization method is the only one,
it is assumed that
a complete rewriting system is already generated using the {\em kb} or
{\em akb} command. 
If the rewriting system being considered is not complete, then an equation is still a theorem in the equational theory
if the {\em prove} declares it to be so; otherwise, we cannot
say that it is not an equational consequence.

In Appendix C, after 
a complete system for free groups is generated,
the {\em prove} command is illustrated on an equation.

If the chosen method is {\em forward reasoning} and the proof by normalization
method fails to prove an equation, then \RRL puts the equation aside
and invokes the Knuth-Bendix completion procedure. 
During the completion,
as long as the current rewriting system
can reduce the equation to identity, \RRL stops and reports the success.
This option has advantage over the proof by normalization method
in that \RRL can still prove theorems in the equational theory
which does not have finite canonical systems.

\subsection{Proof by Refutation: \em kb, refute}

In this approach, the input to \RRL
is either the negation of a formula to be proved or equivalently,
the hypotheses and the negation of the conclusion.
This can also be done by invoking the command {\em refute} which
accepts a list of equations as hypotheses and the last equation as the
conclusion to be proved from the given hypothesis
(the {\em refute} command negates the conclusion).  The Knuth-Bendix
completion procedure is then run.  If an inconsistency is detected,
for example, the rule {\tt true --> false}, {\tt x --> y}, or 
{\tt x --> c}, where {\tt c} is a constant, is generated, then the original
formula is valid.  This approach is mostly used for the first order
predicate calculus; however, it can also be used for equational
formulas. This approach can also be used to check whether a given theory
is inconsistent. 

Two methods of the same approach have been implemented in \ERRL:
Depending on the chosen method, the input formulas are
transformed into either 
(i) polynomials of the boolean ring, then
into (unconditional) rewrite rules, if using the boolean-ring method 
or (ii) conditional rewrite rules, if using the conditional rule
method. Note that for the conditional rule method, the legal input
is either conditional equations or clauses (see Section~\ref{condi}).

As stated above, for making rules from formulas,
a total lexicographic order on predicate symbols
(unless specified differently) is assumed which
is extended to compare monomials (conjunction of atomic formulas)
either by size (default) or by pure lexicographic ordering; arguments
to predicate symbols are compared using {\em rpo} or {\em lrpo}. If the user
wishes to impose a different order on predicate symbols,
it can be done using the {\em operator/precedence} command
before invoking the {\em kb} or {\em refute} commands.

See Appendices F, G and H for examples illustrating different
capabilities of \RRL for theorem proving in predicate calculus.

For general theorem proving, the proof by refutation method
is better than the proof by forward reasoning method because
the negation of the conclusion may help to find the proof quickly.
However, when the input is a set of pure equations and the conjecture
is also a pure equation, it may be better to use
the forward reasoning method because the negation of the conjecture
will force \RRL to run the completion procedure
for the first-order predicate calculus, which is less efficient
than the completion procedure for pure equations.

\subsection{Proof by Inductionless Induction: \em prove} 

In the {\em prove} command, if the
input equation cannot be proved to be a theorem by normalization, 
\RRL prompts the user whether he (or she) want to checks
whether the input equation is an inductive theorem, i.e., for every
instantiation of variables in the equation by ground terms, the
equation is in the equational theory.  If the equation is not an
inductive theorem, \RRL notifies the user of that; it is possible to
obtain from the generated rules the values of variables appearing in
the equation for which the equation does not hold. If the equation is
an inductive theorem, there are two possibilities: 
\begin{enumerate}
\item
\RRL stops with
a complete rewriting system, which implies that the equation is indeed
an inductive theorem and the resulting complete rewriting system is a
decision procedure for the augmented theory, 
\item
conditions similar
to those discussed above when the command {\em kb} is invoked arise.
\end{enumerate}  

\RRL supports two methods of inductionless-induction. One is based on
the ground-reducibility \cite{JK86}, the other is based on
the test set \cite{KNZ86}.  The correctness of the two methods are
proved for the left-linear systems. A theoretical result is known for
an arbitrary rewriting system; it is, however, quite inefficient
and is thus not implemented. Either method can be selected 
using the {\em option/prove} command; the default is to
use the test set method. When the system is not left-linear,
no warning messages will be given by \ERRL; so the user is responsible
in this case for the correctness of the result.
  
For both methods,
identifying a subset of functions as constructors is
necessary; 
if no constructors exist, the user may declare all operators 
are constructors and the methods still work. 
If some operators are non-constructors,
\RRL checks at first
the sufficient completeness of these functions, then uses
the inductionless induction method.

The inductionless induction methods are 
integrated only with the mechanisms to handle
associative and commutative functions. As a result, the methods
may not work in the presence of
functions with other properties like transitivity, cancellation laws,
distribution laws.
  
See Appendix L for an illustration of a proof by inductionless
induction using \ERRL.

\subsection{Proof by the Cover Set Induction: \em prove}

In an algebraic specification of abstract data types,
a function can be defined using a set of equations, which will be referred
as {\em definition equations} of that function.
In addition to the arity declaration of each function,
the cover set method requires the user to distinguish between
definition equations and properties of a function
(previous proved lemmas or theorems).
\RRL recognizes an equation with the token {\tt :=} as a definition
equation of the outermost symbol of the left side of that equation.
For instance, \RRL considers the first two of the following equations
to be the definition of $+$, while the last one is a property of $+$
($0 + x == x$ can be proved to be an inductive theorem if 
0 and $suc$ are the only constructors).
\begin{verbatim}
        [+ : int, int -> int]
        x + 0 := x
        x + suc(y) := suc(x + y)
        0 + x == x
\end{verbatim}
After all the equations are input to \ERRL, the user
may invoke {\em makerule} or {\em kb} to obtain a set of terminating
rewrite rules from the input equations. Note that the result of
the {\em kb} is no long a canonical system even it stops 
successfully. After declaring which operators are constructors,
the user may prove an equation using the command {\em prove}.
\RRL will choose induction scheme automatically, according
to the function symbols appearing in the equation, along with the
definition equations of each function symbol. Currently, \RRL does not
support any methods for checking whether a cover set used for an
induction scheme is complete, i.e., it covers all values (n-tuples of
values) of a data structure on which the induction is being performed.

Once an equation is proved, \RRL saves it in the system and is ready
to prove others. See Appendix M for an illustration of 
a proof by the cover set induction method.

The cover set method 
may fail to prove an equation;
it is a drawback of this method that
it is often not clear why it failed: whether the conjecture being proved is
false or a proof of the conjecture needs additional lemmas.  
The method does not provide any help to the user in the case of failure.
If an important lemma is missing, it is the user's job to figure out
what it is.

\section{Registry of Commands: {\em log, unlog, auto}}

\RRL provides a facility to register user's input or commands into a file, 
which
could be re-executed later.  The command to start registering the user's input
is {\em log} which prompts the user to give a file
name; the suffix {\tt .cmd} is appended to the given name. The
commands {\em unlog} or {\em quit} can be used to close the {\tt cmd} file
and stop registering the input.
  
The commands in a {\tt cmd} file could be 
subsequently read and executed using the command
{\em auto}, giving it a file name.  \RRL takes the input commands from
the file and executes them one after another.
  
Command files thus made are useful for a demonstration, some routine work
like initializing some flags etc., or testing a new change to \RRL using
standard examples.
  
If a file {\tt init-rrl.cmd} is present in the subdirectory where \RRL
is invoked, the commands in the file are executed first.  A use
of {\tt init-rrl.cmd} is that the user could run \RRL on the background
of the UNIX system and get a transcript of the execution at the same
time. Following is a command file made to complete the group
axiomatization with one equation in an automatic mode. The transcript
file is {\em result}.\\ 
\\
\% {\em cat init-rrl.cmd}
\begin{verbatim}
add
(x / ((((x / x) / y) / z) / (((x / x) / x) / z))) == y  
]
akb
quit
\end{verbatim}
\% {\em rrl $>$ result \tt \&}
\begin{verbatim}
[1] 9860
\end{verbatim}

\section{Statistics Information and Trace Facilities}

The command {\em stats} displays various statistics such as times
spent in different operations, including normalization, adding rules,
unification, and termination, as well as total numbers of rules and
critical pairs ever generated, and total time. Whenever the {\em undo}
command is invoked, information about the statistics gets garbled and
cannot be trusted.
  
\RRL supports 3 different levels of trace messages. The default level
value is 2 in which new rules added to the system and existing rules
modified or deleted are displayed.  Level 1 suppresses most of the
trace messages of level 2; it only gives information when the user
needs to make a choice.  At level 3, critical pairs generated and
processed are also displayed in addition to the information given at
level 2. This selection is made using the {\em option/trace}
command.

\section{On-Line Help}
Limited help is available on-line while running \ERRL. The user could
get it at different command levels by invoking the {\em help} command.

\section{Reporting Bugs and other problems}
If the user encounters problems in using \RRL or finds bugs in \ERRL,
please report them to Hantao Zhang or Deepak Kapur. Their mailing
addresses as well as net addresses are given at the beginning of the
manual. Any other feedback as well as information on how 
\RRL is being used will be greatly appreciated.

\subsubsection{Acknowledgements}

\RRL has been designed in collaboration with G. Sivakumar.
Dan Benanav, Chris Connolly, David Cyrluk, Tim Freeman,
Richard Harris, Mark Lopez and Bill Shang also contributed to some
parts of the code. David Cyrluk, David Musser, Paliath Narendran, and
Jonathan Stillman provided valuable suggestions and feedback. John
Guttag's group at MIT and Pierre Lescanne of CRIN, Nancy, also
provided suggestions. Special thanks go to David Cyrluk for his help
in fixing some bugs in \ERRL.

