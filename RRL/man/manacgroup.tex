\chapter{Generating a complete system with ac-operator}
\normalsize
\rm
This transcript illustrates the completion procedure for
equations when some function symbols are interpreted;
in this axiomatization of free abelian groups, the binary function
* is specified using the {\em operator} command to be associative and
commutative. In this case also, the goal is to generate
a complete system. Two different complete systems are generated
by orienting the equation $i(z) * i(z1) == i(z1 * z)$ in two
different ways both of which happen to be terminating.
Each system has different canonical form for expressions.
The termination of the case when the equation is oriented right-to-left
can be shown using \ERRL, whereas the termination
algorithm for associative-commutative rewriting in \RRL is not 
powerful to show the termination in the case
when the equation is oriented left-to-right.
This transcript also illustrates the backtracking (or undo) feature in
\ERRL.

\small
\tt
\begin{verbatim}

Type Add, Akb, Auto, Break, Clean, Delete, Dump, Grammar, Init, Kb, List,
     Log, Makerule, Narrow, Norm, Order, Option, Operator, Prove, Quit, Read,
     Refute, Stats, Suffic, Undo, Unlog, Write or Help.
\end{verbatim}
RRL-> {\em r group}
\begin{verbatim}
Equations read in are:
 1. (e * x) == x  [user, 1]
 2. (i(x) * x) == e  [user, 2]
 3. (x * (y * z)) == ((x * y) * z)  [user, 3]
New constant set is: { e }
Time = 0.09999999999999432 sec
Type Add, Akb, Auto, Break, Clean, Delete, Dump, Grammar, Init, Kb, List,
     Log, Makerule, Narrow, Norm, Order, Option, Operator, Prove, Quit, Read,
     Refute, Stats, Suffic, Undo, Unlog, Write or Help.
\end{verbatim}
RRL-> {\em oper ac *}
\begin{verbatim}
'*' is associative & commutative now.
Time = 0.03333333333333144 sec
Type Add, Akb, Auto, Break, Clean, Delete, Dump, Grammar, Init, Kb, List,
     Log, Makerule, Narrow, Norm, Order, Option, Operator, Prove, Quit, Read,
     Refute, Stats, Suffic, Undo, Unlog, Write or Help.
\end{verbatim}
RRL-> {\em kb}
\begin{verbatim}
----- Step 1 -----
Adding rule:   [1] (x * e) ---> x  [user, 1]
Trying to orient equation: 
  (x * i(x)) == e  [user, 2]
  To prove:  (x * i(x))  >  e
  Here are some precedence suggestions:
       1.  * > e
       2.  i > e
Either type a list of numbers or
Type Abort, Display, Equiv, Twoway, Operator, Status, LR, RL, Postpone, MakeEq,
     Quit, Undo or Help.
\end{verbatim}
RRL>KB> {\em 1 2}
\begin{verbatim}
----- Step 2 -----
Adding rule:   [2] (x * i(x)) ---> e  [user, 2]
Adding rule:   [3] i(e) ---> e  [2, 1]
Adding rule:   [4] i(i(z)) ---> z  [2, 2]
Adding rule:   [5] (z * i((z1 * z))) ---> i(z1)  [2, 2]
Trying to orient equation: 
  (z * i(z1)) == i((z1 * i(z)))  [2, 5]
  To prove:  (z * i(z1))  >  i((z1 * i(z)))
        or:  (z * i(z1))  <  i((z1 * i(z)))
  Here are some precedence suggestions:
       1.  * > i                     2.  i > *
Either type a list of numbers or
Type Abort, Display, Equiv, Twoway, Operator, Status, LR, RL, Postpone, MakeEq,
     Quit, Undo or Help.
\end{verbatim}
RRL>KB> {\em 2}
\begin{verbatim}
----- Step 3 -----
Adding rule:   [6] i((z * i(z1))) ---> (z1 * i(z))  [2, 5]
Adding rule:   [7] i((z * z1)) ---> (i(z) * i(z1))  [2, 5]
  Deleting rule:  [5] (z * i((z1 * z))) ---> i(z1)  [2, 2]
  Deleting rule:  [6] i((z * i(z1))) ---> (z1 * i(z))  [2, 5]
Your system is canonical.
  [1] (x * e) ---> x  [user, 1]
  [2] (x * i(x)) ---> e  [user, 2]
  [3] i(e) ---> e  [2, 1]
  [4] i(i(z)) ---> z  [2, 2]
  [7] i((z * z1)) ---> (i(z) * i(z1))  [2, 5]
Type Add, Akb, Auto, Break, Clean, Delete, Dump, Grammar, Init, Kb, List,
     Log, Makerule, Narrow, Norm, Order, Option, Operator, Prove, Quit, Read,
     Refute, Stats, Suffic, Undo, Unlog, Write or Help.
\end{verbatim}
RRL-> {\em undo}
\begin{verbatim}
----- Return to Step 2 -----
Equations:
 1. (z * i(z1)) == i((z1 * i(z)))  [2, 5]
Rules:
  [1] (x * e) ---> x  [user, 1]
  [2] (x * i(x)) ---> e  [user, 2]
  [3] i(e) ---> e  [2, 1]
  [4] i(i(z)) ---> z  [2, 2]
  [5] (z * i((z1 * z))) ---> i(z1)  [2, 2]
Time = 0.05000000000001137 sec
Type Add, Akb, Auto, Break, Clean, Delete, Dump, Grammar, Init, Kb, List,
     Log, Makerule, Narrow, Norm, Order, Option, Operator, Prove, Quit, Read,
     Refute, Stats, Suffic, Undo, Unlog, Write or Help.
\end{verbatim}
RRL-> {\em kb}
\begin{verbatim}
----- Step 3 -----
Trying to orient equation: 
  (z * i(z1)) == i((z1 * i(z)))  [2, 5]
  To prove:  (z * i(z1))  >  i((z1 * i(z)))
        or:  (z * i(z1))  <  i((z1 * i(z)))
  Here are some precedence suggestions:
       1.  * > i                     2.  i > *
Either type a list of numbers or
Type Abort, Display, Equiv, Twoway, Operator, Status, LR, RL, Postpone, MakeEq,
     Quit, Undo or Help.
\end{verbatim}
RRL>KB> {\em 1}
\begin{verbatim}
----- Step 4 -----
Sorry did not work, try again.
  To prove:  (z * i(z1))  >  i((z1 * i(z)))
        or:  (z * i(z1))  <  i((z1 * i(z)))
I have no precedence suggestions.  
Try doing Equiv or Status.
Type Abort, Display, Equiv, Twoway, Operator, Status, LR, RL, Postpone, MakeEq,
     Quit, Undo or Help.
\end{verbatim}
RRL>KB> {\em undo}
\begin{verbatim}
----- Return to Step 3 -----
Trying to orient equation: 
  (z * i(z1)) == i((z1 * i(z)))  [2, 5]
  To prove:  (z * i(z1))  >  i((z1 * i(z)))
        or:  (z * i(z1))  <  i((z1 * i(z)))
  Here are some precedence suggestions:
       1.  * > i                     2.  i > *
Either type a list of numbers or
Type Abort, Display, Equiv, Twoway, Operator, Status, LR, RL, Postpone, MakeEq,
     Quit, Undo or Help.
\end{verbatim}
RRL>KB> {\em rl}
\begin{verbatim}
Orientation of equations being done manually.
!!!!!! Warning: Rewriting may not terminate !!!!
!!!!!! Warning: Resulting system will be locally-confluent;
         But it may not be canonical  !!!!
\end{verbatim}
Is it ok to continue ? (y,n) {\em y}
\begin{verbatim}
----- Step 4 -----
Adding rule:   [6] i((z * i(z1))) ---> (z1 * i(z))  [2, 5]
Trying to orient equation: 
  i((z * z1)) == (i(z) * i(z1))  [4, 6]
  To prove:  i((z * z1))  >  (i(z) * i(z1))
        or:  i((z * z1))  <  (i(z) * i(z1))
  Here are some precedence suggestions:
       1.  i > *                     2.  * > i
Either type a list of numbers or
Type Abort, Display, Equiv, Twoway, Operator, Status, LR, RL, Postpone, MakeEq,
     Quit, Undo or Help.
\end{verbatim}
RRL>KB> {\em rl}
\begin{verbatim}
----- Step 5 -----
Adding rule:   [7] (i(z) * i(z1)) ---> i((z * z1))  [4, 6]
Your system is locally confluent.
  [1] (x * e) ---> x  [user, 1]
  [2] (x * i(x)) ---> e  [user, 2]
  [3] i(e) ---> e  [2, 1]
  [4] i(i(z)) ---> z  [2, 2]
  [5] (z * i((z1 * z))) ---> i(z1)  [2, 2]
  [6] i((z * i(z1))) ---> (z1 * i(z))  [2, 5]
  [7] (i(z) * i(z1)) ---> i((z * z1))  [4, 6]
\end{verbatim}
\rm
\normalsize
