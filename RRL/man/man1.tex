\pagenumbering{roman}
\tableofcontents
\chapter{Introduction}
\pagenumbering{arabic}

\section{What is \ERRL?}

\RRL is an interactive reasoning program based on rewriting techniques,
with many primitives and algorithms
for experimenting with existing automated reasoning algorithms 
as well as for developing new reasoning algorithms
\cite{KapurSiva,KZ881}.  Reasoning
algorithms are used in many application areas in computer science and
artificial intelligence, in particular, analysis of specifications of
abstract data types, verification of hardware and software, synthesis
of programs from specifications, automated theorem proving, expert
systems, deductive data basis, planning and recognition, logic
programming, constraint satisfaction, natural language processing,
knowledge-based systems, robotics, vision.  

The project for developing rewrite rule laboratories was jointly
conceived in 1983 by the first author and David Musser of Rensselaer
Polytechnic Institute (RPI) and John Guttag of MIT when rewrite
methods were gaining attention because of their use in the analysis of
algebraic specifications of user-defined data types.  The work on the
\RRL project was started at General Electric Corporate Research and
Development (GECRD) and RPI in Fall 1983, following a workshop on
rewrite rule laboratories in Schenectady, New York
\cite{Guttagetal84}.  Since then, \RRL has continued to evolve.
\RRL has been designed by the authors in collaboration with G. Sivakumar.
Most of the code in \RRL has been written by G. Sivakumar 
(1983--1985) and
the second author (1985--present). 
\RRL is written in a subset of Zeta Lisp that is (almost) compatible
with Franz Lisp and is more than 17,000 lines of code.  In order to
make \RRL portable and available for wider distribution, conscious
attempt was made not to use special features of Zeta Lisp and the Lisp
machine environment.
 
\RRL was initially used to study existing algorithms, including the
Knuth-Bendix completion procedure \cite{KnuthBendix}, and
Dershowitz's algorithm for recursive path ordering \cite{Dershowitz87}
to orient equations into terminating rewrite
rules \cite{KapurSiva}.  However, this environment soon began serving as a
vehicle for generating new ideas, concepts and algorithms which could
be implemented and tested for their effectiveness.  Experimentation
using \RRL has led to development of new heuristics for decision
procedures and semi-decision procedures in theorem proving
\cite{KMN84,ZK89}, complexity studies of primitive operations
used in \RRL \cite{KN87}, and new approaches for first-order theorem
proving \cite{KN85,ZK88,Zhang88}, proofs by induction
\cite{KNZ86,ZKM88,Zhang88}.

\RRL has been distributed to over 20 universities and research
laboratories. Any one interested in obtaining a copy of \RRL can write
to the authors.

\section{What can \RRL do?}

\RRL currently
provides facilities for
\begin{itemize}
\item 
automatically proving theorems in first-order predicate calculus
with equality,
\item
generating decision procedures for first-order (equational)
theories using completion procedures,
\item 
checking the consistency and completeness of equational
specifications, and
\item 
proving theorems by 
induction using different approaches - methods
based on the {\em proof by consistency} approach \cite{Musser80,KapurMusser84}
(also called the {\em inductionless-induction} approach) as well as 
the explicit induction approach.
\end{itemize}

The input to \RRL is a first-order theory specified by a finite set of
axioms.  Every axiom is first transformed into an equation (or a
conditional equation).  The kernel of \RRL is the {\em extended
Knuth-Bendix completion procedure} \cite{KnuthBendix,LB77,PS81} which
first produces terminating rewrite rules from equations and then
attempts to generate a canonical rewriting system for the theory
specified by a finite set of formulas.  If \RRL is successful in
generating a canonical set of rules, these rules associate a unique
normal form for each congruence class in the congruence relation
induced by the theory.  These rules thus serve as a decision procedure
for the theory.  

For proving a single first-order formula, \RRL also supports the
proof-by-contradiction (refutational) method. The set of hypotheses
and the negation of the conclusion are given as the input, and \RRL
attempts to generate a contradiction, which is a system including the
rule $true \rightarrow false$.

New algorithms and methods for automated reasoning have been developed
and implemented in \ERRL. 
\RRL supports an implementation of a first-order theorem proving method
based on polynomial representation of formulas using the exclusive-or
connective as `+' and the and connective as `.' \cite{Hsiang85}, and
using \Groebner basis like completion procedure \cite{KN85}.  Its
performance compares well with theorem provers based on resolution
developed at Argonne National Lab as well as theorem provers based on
the connection graph approach developed at SRI International and University of
Kaiserslautern. Most of the hardware verification work mentioned later
in the section on applications of \RRL was done using this method of
theorem proving.
 
Recently, another method for automatically proving theorems in
first-order predicate calculus with equality using conditional
rewriting has also been implemented and successfully tried on a number
of problems \cite{ZK88}.  This method combines ordered resolution and
oriented paramodulation of Hsiang and Rusinowitch \cite{HR86} into a
single inference rule, called {\it clausal superposition}, on
conditional rewrite rules. These rewrite rules can be used to simplify clauses
using conditional rewriting.

\RRL has two different methods implementing
the proof by consistency approach for proving
properties by induction: the {\em test set} method developed in
\cite{KNZ86} as well as the method using {\em ground-reducibility}
(or {\em quasi-reducibility})
developed in \cite{JK86}.  These methods can be used for checking
structural properties of equational specifications such as the {\em
sufficient-completeness} \cite{KNZ85} and consistency properties
\cite{Musser80,KapurMusser84}.

Recently, a method based on the concept of a {\it cover set} of a
function definition has been developed for mechanizing proofs by
explicit induction for equational specifications \cite{ZKM88,Zhang88}.  The
method is closely related to Boyer and Moore's approach.  It
has been successfully used to prove over a hundred theorems about
lists, sequences, and number theory including the unique prime
factorization theorem for numbers; see \cite{ZKM88,Zhang88}.
 
\RRL supports different strategies for normalization and generation of
critical pairs and other features. These strategies play a role
similar to the various strategies and heuristics available in
resolution-based theorem provers. \RRL provides many options to the
user to adapt it to a particular application domain. Because of
numerous options and strategies supported by \ERRL, it can be used as a
laboratory for testing algorithms and strategies. 

\section{Some Applications of \RRL}
 
\RRL has been successfully used to attack difficult problems
considered a challenge for theorem provers.  An automatic proof of a
theorem that an associative ring in which every element $x$ satisfies
$x^3 = x,$ is commutative, was obtained using \RRL in nearly 2 minutes
on a Symbolics machine.  Previous attempts to solve this problem using
the completion procedure based approach required over 10 hours of
computing time \cite{Stickel84}.  With some special heuristics
developed for the family of problems that for $n > 1$, for every $x$,
$x^n = x$ implies that an associative ring is commutative, \RRL takes
5 seconds for the case when $n$ = 3, 70 seconds for $n$ = 4, and 270
seconds for $n$ = 6 \cite{KMN84,ZK89}.  To our knowledge, this is the
first computer proof of this theorem for $n$ = 6. Using algebraic
techniques, we have been able to prove this theorem for many even
values of $n$. \RRL has also been used to prove the equivalence of
different non-classical axiomatizations of free groups developed by
Higman and Neumann to the classical three-law characterization
\cite{KZ882}.
 
\RRL can be also used as a tool for demonstrating approaches towards
specification and verification of hardware and software.  \RRL is a
full-fledged theorem prover supporting different methods for
first-order and inductive theorem proving in a single reasoning
system.  It has been successfully used at GECRD for hardware
verification. Small combinational and sequential circuits, leaf cells
of a bit-serial compiler have been verified \cite{NS881}. The most
noteworthy achievement has been the detection of 2 bugs in a VHDL
description of a 10,000 transistor chip implementing Sobel's edge
detection algorithm for image processing and after fixing those bugs,
verifying the chip description to be consistent with its behavioral
description \cite{NS882}.

\RRL has also been used in teaching a course on theorem proving
methods based on rewrite techniques, 
providing students with an opportunity to try methods and
approaches discussed in the course.  It has served as a basis for
class projects and thesis research. 

Additional details about the capabilities of \RRL and the associated theory
as well as possible applications of \RRL can be found in the 
papers listed in the selected bibliography at the end of this manual.

\section{Organization of the Manual}
  
This manual is intended to introduce the reader to \RRL as a theorem
prover. It assumes familiarity with the rewriting approach to theorem
proving.  The reader may wish to consult the literature on the
rewriting paradigm \cite{Huet80,Bundy,RTA85,JSC}.  In the next
chapter, we outline the proof methods supported by \ERRL; they are
mainly based on completion and rewriting concepts.  In Chapter III, we
group the commands of \RRL by their functionality and describe what
each command does.

In order to help someone not very familiar with the rewriting
paradigm, we have included transcripts of a number of examples which
were run on \ERRL. These examples illustrate the capabilities and
features of \RRL as a theorem prover.  The rest of the manual should be
read along with these transcripts reproduced in the Appendices. Below,
we list the examples discussed in the Appendices along with one
sentence description.  A more detailed description is given at the
beginning of the transcript in each Appendix.  The best way to
familiarize with \RRL might be to compare what the user would like to
do with the examples in the Appendices, pick the one which closely
matches with the user's needs, and follow the commands in the
transcript of that example.

Appendices C, D, and E illustrate how \RRL can be used to generate a
complete rewriting system for an equational theory.  
As stated above, such a complete
system is a decision procedure for the equational theory; in order to
check whether an equation is a theorem or not, one just needs to
normalize the two sides of the equation using this complete system and
check whether the results are the same. Appendix D is also an
illustration of how \RRL can be used to check the equivalence of two
different axiomatization of the same theory.

Appendices F, G and H illustrate \RRL as a refutational theorem
prover. \RRL can also be used to generate a complete system from a set
of axioms (if the \RRL terminates, which is guaranteed for
propositional calculus) which serves as a decision procedure for these
axioms. 
Appendix F illustrates the use of rewriting methods
for handling equality. 
Appendix G includes three examples by the boolean ring method:
one for propositional calculus; the second for
first-order predicate calculus
without equality and the third for the use
 of paramodulation
in this method.
Appendix H illustrates the
use of conditional rewrite rules for first order predicate
calculus with the equality.

Appendix I illustrates the use of narrowing for solving an equation
modulo a theory. This can be used to support logic programming.

Appendix J illustrates the use of \RRL for checking the sufficient
completeness property of an abstract data type specified using
equations. In case that a function specification is not complete, \RRL
proposes templates on which the function needs to be specified for it
to be completely specified.

Appendix K illustrates the use of \RRL as an interpreter for
equational programs written in a functional style.

Appendixes L and M illustrate the proof-by-consistency method
and the cover set induction method for proving
equations by induction.

