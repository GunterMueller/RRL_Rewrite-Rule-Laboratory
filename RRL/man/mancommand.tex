\chapter{RRL Commands}
\small
\rm
Currently you can do the following:

\begin{tabular}{ll}
{\em Add} & --- Input equations from the terminal.\\
{\em Akb} & --- Run the automatic Knuth-Bendix completion procedure. \\
{\em Auto} & --- Automatically execute a sequence of commands in a `cmd' file.\\
{\em Break} & --- Talk to Lisp directly.\\
{\em Clean} & --- Clean the history stack to save space.\\
{\em Delete} & --- Delete a list of equations or rules.\\
{\em Grammer} & --- Display the input grammer of RRL.\\
{\em History} & --- Put a copy of the current state into the history stack.\\
{\em Init}   & --- Initialize RRL.\\
{\em Kb}     & --- Run the Knuth-Bendix Completion Procedure.\\
{\em List}   & --- List the current set of equations and rules.\\
{\em Load}   & --- Load a system saved in a file by the {\em save} command.\\
{\em Log}    & --- Register all commands in a file (to be used by {\em auto}).\\
{\em Makerule} & --- Orient equations into rewrite rules without superposition. \\
{\em Norm}   & --- Using different strategies to normalize a term.\\
{\em Option} & ---  Set RRL flags and parameters.\\
{\em Operator} & --- Set properties of operators (precedence, constructors, etc.).\\
{\em Prove}  & --- Prove an equation by rewriting or by induction.\\
{\em Quit}   & --- Quit RRL.\\
{\em Read}   & --- Input equations from a file.\\
{\em Refute} & --- Read a set of equations and negate the last one for refutation.\\
{\em Save}   & --- Save the current state of RRL into a file. \\
{\em Stats}  & --- List the current state of RRL. \\
{\em Suffice} & --- Test sufficient completeness.\\
{\em Undo}   & --- Undo RRL to the last user's interaction.\\
{\em Unlog}  & --- Stop registrying commands into a file. \\
{\em Write}  & --- Write equations and rules to a file.\\
{\em Help}   & --- Give this message. 
\end{tabular} 

Note: One does not need to type the whole command name, a 
substring will suffice except
that the first letter must be given. If the typed substring matches more
than one command, the first one is chosen. For example, when you 
give {\em o} or {\em or}, RRL takes it for {\em operator}.
When you give {\em od}, RRL takes it for {\em order}..
No distinction is made between uppercase and lowercase characters.
\normalsize
\rm
