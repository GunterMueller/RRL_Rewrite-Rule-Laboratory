\chapter{RRL as an interpreter for functional definitions}
\normalsize
\rm
We give below a recursive definition of Ackermann's function.  The
termination of this definition can be easily verified using the {\em
makerule} command. Further, it can also be checked using {\em suffice}
command that the function is completely defined.  \RRL gives a warning
that the rewriting system obtained after invoking {\em makerule} may
not be canonical. \RRL subsequently asks for the constructors of the
domain on which the function is defined, which in this case, are {\em
0, s}. If \RRL finds a function to be completely defined, then the
function is indeed completely defined even if the rewriting system is
not canonical.  That portion of the transcript was generated using
extended trace to exhibit how \RRL checks for the completeness of the
definition using the test set method.  Later, we exhibit how the {\em
norm} command can be used to compute the value of Ackermann's function
on a particular input.

To use \RRL as an interpreter for a functional language, the user can
\begin{enumerate}
\item give the functional definition as input, 
\item invoke {\em makerule} to make
rewrite rules from the functional definition,
\item check for the completeness of the definition, if desired, using
{\em suffic} and by giving the constructors of the domain on which the function
is defined, and finally
\item use {\em norm} to execute the function by reading the term corresponding
to the functional invocation and normalizing it.
\end{enumerate}

\small
\tt
\begin{verbatim}
Type Add, Akb, Auto, Break, Clean, Delete, Grammar, History, Init, Kb, List,
     Load, Log, Makerule, Narrow, Norm, Option, Operator, Prove, Quit, Read,
     Refute, Save, Statics, Suffice, Undo, Unlog, Write or Help.
\end{verbatim}
RRL-> {\em read ackermann}
\begin{verbatim}

New operators have the arities:
      0: -> int
      s: int -> int
      ack: int, int, int -> int
New constant set is: { 0 }
Equations read in are:
 1. ack(0, 0, x) == x  [user, 1]
 2. ack(0, s(x), y) == s(ack(0, x, y))  [user, 2]
 3. ack(s(0), 0, y) == 0  [user, 3]
 4. ack(s(s(x)), 0, y) == s(0)  [user, 4]
 5. ack(s(z), s(x), y) == ack(z, ack(s(z), x, y), y)  [user, 5]
New constant set is: { 0 }

Type Add, Akb, Auto, Break, Clean, Delete, Grammar, History, Init, Kb, List,
     Load, Log, Makerule, Narrow, Norm, Option, Operator, Prove, Quit, Read,
     Refute, Save, Statics, Suffice, Undo, Unlog, Write or Help.
\end{verbatim}
RRL-> {\em makerule}
\begin{verbatim}
Adding rule:   [1] ack(0, 0, x) ---> x  [user, 1]
Trying to orient equation: 
  ack(0, s(x), y) == s(ack(0, x, y))  [user, 2]
  To prove:  ack(0, s(x), y)  >  s(ack(0, x, y))
        or:  ack(0, s(x), y)  <  s(ack(0, x, y))
  Here are some precedence suggestions:
       1.  ack > s                   2.  s > ack
       3.  ack > 0                   4.  0 > ack
       5.  0 > s                     6.  s > 0
Either type a list of numbers or
Type Abort, Display, Equiv, Twoway, Operator, Status, LR, RL, Postpone, MakeEq,
     Quit, Undo or Help.
\end{verbatim}
RRL>KB> {\em st ack lr}
\begin{verbatim}
Operator, ack, given status: lr
----- Step 1 -----
  To prove:  ack(0, s(x), y)  >  s(ack(0, x, y))
        or:  ack(0, s(x), y)  <  s(ack(0, x, y))
  Here are some precedence suggestions:
       1.  ack > s                   2.  s > ack
       3.  ack > 0                   4.  0 > ack
       5.  0 > s                     6.  s > 0
Either type a list of numbers or
Type Abort, Display, Equiv, Twoway, Operator, Status, LR, RL, Postpone, MakeEq,
     Quit, Undo or Help.
\end{verbatim}
RRL>KB> {\em 1 3}
\begin{verbatim}
Adding rule:   [2] ack(0, s(x), y) ---> s(ack(0, x, y))  [user, 2]
Adding rule:   [3] ack(s(0), 0, y) ---> 0  [user, 3]
Adding rule:   [4] ack(s(s(x)), 0, y) ---> s(0)  [user, 4]
Adding rule:   [5] ack(s(z), s(x), y) ---> ack(z, ack(s(z), x, y), y)  [user, 5]

Type Add, Akb, Auto, Break, Clean, Delete, Dump, Grammar, Init, Kb, List,
     Log, Makerule, Narrow, Norm, Order, Option, Operator, Prove, Quit, Read,
     Refute, Stats, Suffic, Undo, Unlog, Write or Help.
\end{verbatim}
RRL-> {\em opt tr 3}
\begin{verbatim}

Type Add, Akb, Auto, Break, Clean, Delete, Grammar, History, Init, Kb, List,
     Load, Log, Makerule, Narrow, Norm, Option, Operator, Prove, Quit, Read,
     Refute, Save, Statics, Suffice, Undo, Unlog, Write or Help.
RRL-> operator constructor
Type operators you wish to be constructors: 0 s
Is 's' a free constructor ? (y,n,yes,no) y

Type Add, Akb, Auto, Break, Clean, Delete, Grammar, History, Init, Kb, List,
     Load, Log, Makerule, Narrow, Norm, Option, Operator, Prove, Quit, Read,
     Refute, Save, Statics, Suffice, Undo, Unlog, Write or Help.
\end{verbatim}
\tt RRL-> \em option trace 3\tt
\begin{verbatim}
Type Add, Akb, Auto, Break, Clean, Delete, Dump, Grammar, Init, Kb, List,
     Log, Makerule, Narrow, Norm, Order, Option, Operator, Prove, Quit, Read,
     Refute, Stats, Suffic, Undo, Unlog, Write or Help.
\end{verbatim}
RRL-> {\em suffice}
\begin{verbatim}
The system has the following constructors:
     Type 'int': { 0, s }
Nmuber of Candidates = 2
Add to test set: 0
Add to test set: s(x1)
Time spent in computing TestSet  =  0.0 sec
The schemas of the term
    ack(x1, x2, x3)
    are
    ack(0, x2, x3)
    ack(s(x1), x2, x3)
The schemas of the term
    ack(0, x2, x3)
    are
    ack(0, 0, x3)
    ack(0, s(x1), x3)
The schemas of the term
    ack(s(x1), x2, x3)
    are
    ack(s(0), x2, x3)
    ack(s(s(x1)), x2, x3)
The schemas of the term
    ack(s(0), x2, x3)
    are
    ack(s(0), 0, x3)
    ack(s(0), s(x1), x3)
The schemas of the term
    ack(s(s(x1)), x2, x3)
    are
    ack(s(s(x1)), 0, x3)
    ack(s(s(x1)), s(x11), x3)
Specification of 'ack' is completely defined.
Processor time used      =  0.1333333333333337 sec

Type Add, Akb, Auto, Break, Clean, Delete, Dump, Grammar, Init, Kb, List,
     Log, Makerule, Narrow, Norm, Order, Option, Operator, Prove, Quit, Read,
     Refute, Stats, Suffic, Undo, Unlog, Write or Help.
\end{verbatim}
RRL-> {\em norm}
\begin{verbatim}
Input a term in the form T  or T if C :
\end{verbatim}
{\em ack(0, 0, 0)}
\begin{verbatim}
The term read in is: ack(0, 0, 0)
\end{verbatim}
Type abort, norm, list, read, strat, quit, help --> {\em n}
\begin{verbatim}
The term to be normalized is: ack(0, 0, 0)
The normal form of the  term is: 0
Processor time used = 0.0 sec
\end{verbatim}
Type abort, norm, list, read, strat, quit, help --> {\em r}
\begin{verbatim}
Input a term in the form T or T if C :
\end{verbatim}
{\em ack(0, s(s(0)), s(s(s(0))))}
\begin{verbatim}
The term read in is:  ack(0, s(s(0)), s(s(s(0))))
\end{verbatim}
Type abort, norm, list, read, strat, quit, help --> {\em n}
\begin{verbatim}
The term to be normalized is: ack(0, s(s(0)), s(s(s(0))))
The normal form of the term is: s(s(s(s(s(0)))))
\end{verbatim}
Type abort, norm, list, read, strat, quit, help --> {\em r}
\begin{verbatim}
Input a term in the form T  or T if C :
\end{verbatim}
{\em ack(s(s(s(0))), s(s(0)), s(s(s(0))))}
\begin{verbatim}
The term read in is:  ack(s(s(s(0))), s(s(0)), s(s(s(0))))
\end{verbatim}
Type abort, norm, list, read, strat, quit, help --> {\em n}
\begin{verbatim}
The term to be normalized is: ack(s(s(s(0))), s(s(0)), s(s(s(0))))
The normal form of the term is:
s(s(s(s(s(s(s(s(s(s(s(s(s(s(s(s(s(s(s(s(s(s(s(s(s(s(s(0)))))))))))))))))))))))))))
Processor time used = 1.333333333333333 sec
\end{verbatim}
\rm
\normalsize
